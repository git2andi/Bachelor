\chapter{Introduction}~\label{ch:introduction}

In today's rapidly evolving digital landscape, the demand for 3D models continues to grow, driven by the ever-increasing need for immersive and realistic visual experiences. Generative AI techniques have proven to be a transformative force, enabling researchers and practitioners to develop innovative methods that automate the complicated process of creating 3D models. These remarkable innovations can reshape our digital interactions by enabling advanced simulations, in-depth analysis and captivating visualizations of complex real-world phenomena.

Starting out in 3D synthesis can be a challenging experience, particularly for novices with limited prior knowledge. While directly applying the models outlined in Chapter~\ref{ch:models} may appear straightforward, acquiring a more in-depth understanding of the diverse methodologies and their foundational principles greatly enriches the learning process. This deeper understanding not only improves the practical application of these models, but also enables future advances in the field of automatic 3D model generation.

This thesis provides a comprehensive examination into various models and technologies in automated 3D model generation, offering a detailed analysis of their mechanisms, capabilities, and limitations. The study focuses on evaluating the technologies' effectiveness, emphasizing their proficiency in creating functionally robust and aesthetically appealing 3D models. This research contributes to the fields of computer graphics and artificial intelligence, serving as a valuable resource for novices in automatic 3D model generation and inspiring future research.

To provide a foundation for this exploration, a comprehensive examination of the fundamentals of 2D generative AI is undertaken. This includes a comprehensive understanding of key generative models such as Variational Autoencoders (VAEs) \citep{kingmaVAE,rezendeVAE}, Generative Adversarial Networks (GANs) \citep{goodfellowGAN} and Diffusion Models \citep{yangdiffusionSummary,hoDDPMs, sohlDDPM}. Additionally, a brief introduction to Contrastive Language-Image Pre-training (CLIP) \citep{radfordCLIP}, Stable Diffusion \citep{rombachStableDiffusion} and Multilayer Perceptron (MLP) is given. The chapter concludes with an overview of various forms of 3D representation, including Meshes, Point-clouds, Voxels, Neural Radiance Fields (NeRFs) \citep{mildenhallNERF}, Deep Marching Tetrahedra (DMTets) \citep{shen2021DMTet}, and Instant Neural Graphics Primitives (InstantNGPs) \citep{M_ller_2022}.

This research further evaluates various approaches to generating 3D models based on both images and text input. The methods examined include DreamFusion \citep{pooleDreamfusion}, Magic3D \citep{lin2023magic3d}, Fantasia3D \citep{chen2023fantasia3d}, Magic123 \citep{qian2023magic123} and Wonder3D \citep{long2023wonder3d}. Each method presents unique challenges and opportunities, and their results are thoroughly examined through a comparative study. 

The evaluation involves the application of various metrics covering a spectrum of critical aspects, including the accuracy of the input and results, to ensure that the generated models accurately match the input and maintain the intended properties. Also, the level of detail of these models is examined to assess their ability to capture complicated features and nuances. An important criterion is texture realism, i.e.~the ability of the generated models to reflect the authenticity of their real-life counterparts. Furthermore, technical aspects such as symmetry and model integrity are also examined to check the structural soundness and coherence of the generated 3D models using a novel Project named Evaluate3D. This comprehensive analysis leads to an understandable comparison of the strengths and weaknesses of the individual methods examined.

In addition, this thesis casts a discerning eye towards the evolving landscape of 3D modeling and highlights emerging trends that have the potential to reshape the field. These trends include innovations such as video-to-3D methods that open innovative dimensions in the creation of three-dimensional scenes. Also, the important topic of inherited biases is discussed, highlighting the need for more in-depth research to ensure the fairness of these methods. By exploring these unexplored areas, the aim is to promote a fairer and more progressive future for the field of 3D modeling.