\chapter{Basics}\label{ch:basics}

The chapter provides the groundwork necessary for the comparative analysis of automatic 3D model generation techniques by introducing the key technologies that drive these methods. It is essential to have a common understanding of the 2D generative models and 3D data representations that form the basis of this field.

The chapter begins by offering a brief overview of the most prevalent text-to-2D generative models, as they play a pivotal role in the process of 3D synthesis. This includes Variational Autoencoders (VAEs) \citep{kingmaVAE,rezendeVAE}, which are fundamental in providing probabilistic frameworks for learning complex data representations. Following this, Generative Adversarial Networks (GANs) \citep{goodfellowGAN} are introduced, emphasizing their unique training mechanics involving both generator and discriminator components. Additionally, the chapter explores the realm of Diffusion Models, with a specific focus on Denoising Diffusion Probabilistic Models (DDPMs) \citep{hoDDPMs,sohlDDPM}, while also touching upon Stochastic Gradient Methods (SGMs) \citep{song2019SGM} and Stochastic Differential Equations (SDEs) \citep{song2020score,song2021maximum}.

\begin{figure}[H]
  \centering
  \includegraphics[width=.4\columnwidth]{figures/BasicTrilemma.png}
  \caption{The Generative Learning Trilemma: Balancing Quality, Speed, and Diversity in Generative Models.~\citep{xiao2022tackling}}~\label{fig:generativeTrilemma}
\end{figure}

The figure above from \citeauthor{xiao2022tackling} illustrates the ``generative learning trilemma'', effectively capturing the trade-offs among high-quality sample generation, fast sampling, and mode coverage/diversity in these models. Generative Adversarial Networks are noted for their fast and high-quality samples, Denoising Diffusion Models excel in mode coverage/diversity and sample quality, and Variational Autoencoders balance between fast sampling and diversity in their outputs.

Furthermore, the chapter delves into Contrastive Language-Image Pre-training (CLIP) \citep{radfordCLIP}, illustrating its effectiveness in bridging the gap between natural language and visual data, which is crucial for text-guided 3D model generation. The final part of the chapter is devoted to examining various forms of 3D data representation, such as Meshes, Point-Clouds, and Voxels. Understanding these representations is key to comprehending the structural makeup of 3D objects in computational settings.

//TODO check Final Diffusion Model

\section{Variational Autoencoders - VAEs}
\label{VAEs}

Based on the seminal work by \citeauthor{diggleImplicitPrescribed}, generative models can be classified into two broad categories. Prescribed models employ a well-defined, often parametric, mathematical expression for the probability density function (pdf), which enables easier analytical interpretation of the distributions. In contrast, implicit models synthesize new data samples without relying on an explicit pdf, approximating the underlying data distribution on which they were trained \citep{diggleImplicitPrescribed}. Variational Autoencoders (VAEs), which inherit the foundational architecture of autoencoders, belong to the prescribed models category as they require an explicit formulation of the probability density function (pdf) to function effectively. This feature makes VAEs suitable for tasks that require not only the generation but also the understanding of complex data distributions. \citep{kingmaVAE,rezendeVAE,GoodfellowDeepLearning}. Generative Adversarial Networks (GANs) \citep{goodfellowGAN}, discussed later, are a prime example ofthe latter category.

VAEs are essentially based on the architecture of autoencoders, which consist of an encoder and a decoder. The encoder aims to transform the input data into a low-dimensional latent space representation, commonly referred to as a "code" or "bottleneck" \citep{hintonCode, GoodfellowDeepLearning}. This code captures the most relevant features of the input while reducing its dimensionality. Then, the decoder attempts to reconstruct the original input from the obtained latent vector using a loss function. As explained by Goodfellow et al, an autoencoder that only succeeds in copying the exact representation of the input data does not itself prove useful. The essence of autoencoders lies in their ability to copy approximately rather than perfectly, which forces the model to prioritize which aspects of the input to copy \citep{GoodfellowDeepLearning}. This strategic approach often directs autoencoders to "learns useful properties of the data" \citep{GoodfellowDeepLearning}. To summarize, he main goal of an autoencoder is not the reconstruction itself, but the extraction of a meaningful latent vector that serves as a simplified representation of the input data.

The ability to reduce dimensionality has practical implications for improving the efficiency of classification tasks by reducing computational and memory overhead \citep{GoodfellowDeepLearning}. When paired with information retrieval, this dimensionality reduction makes searching in certain low-dimensional spaces particularly efficient \citep{GoodfellowDeepLearning}. Despite these advantages, traditional autoencoders are not designed to generate new data; their main function is to copy and reconstruct the given input \citep{GoodfellowDeepLearning}.

\begin{figure}[ht]
    \centering
      \hspace{.8cm}
      \includegraphics[width=.7\columnwidth]{figures/Autoencoder.png}
      \caption{Autoencoder: The encoder reduces the input dimension to a latent vector that captures the most important features. The decoder then uses this vector to reconstruct the input, with training aimed at minimizing reconstruction loss.}
      \label{fig:figureAE}
    \end{figure}


In variational autoencoders (VAEs), the encoder compresses the input into a space of latent variables and forms a probability distribution over these latent variables called \( q(z|x) \). This distribution helps with robustness against overfitting and enables the model to synthesize new analog data points. During the training phase, VAEs recognize specific regions in the latent variable space, similar to "pools," for different categories of data, allowing for a structured approach to data representation. Unlike standard autoencoders, where it is unclear where a useful latent vector can be sampled for the decoder, VAEs overcome this problem by restricting the latent space to known regions from which vectors can be safely sampled, as cited in \citep{doerschVAE}.

The procedure begins by selecting a sample \( z \) from a code distribution defined by the model, denoted \( p_{model}(z) \). This sample \( z \) is then passed through a differentiable generator network \( g(z) \). A sample \( x \) is then drawn from the distribution \( p_{model}(x; g(z)) \), where its properties are shaped by the processed \( z \) \citep{GoodfellowDeepLearning}. During the training phase, an approximate inference network, also called an encoder \( q(z|x) \), is used to infer \( z \) from \( x \), while \( p_{model}(x|z) \) works as a decoder network to reconstruct \( x \) from \( z \) \citep{GoodfellowDeepLearning}. The main training objective is embodied in the formula:
        
\begin{align}
  L(q) &= \mathbb{E}_{z \sim q(z|x)} \log p_{model}(z, x) + H(q(z|x)) \\
  &= \mathbb{E}_{z \sim q(z|x)} \log p_{model}(x|z) - D_{KL}(q(z|x) || p_{model}(z)) \\
  &\leq \log p_{model}(x)
\end{align}
        
Here, \( L(q) \) acts as a scorecard to evaluate the performance of UAE. The first term \( \log p_{\text{model}}(x|z) \) evaluates how well the UAE can fill in the details to recover the original input, while the second term \( D_{\text{KL}}(q(z|x) || p_{\text{model}}(z)) \) evaluates the complexity of the VAE representation compared to the original term, aiming for simplicity, as mentioned in \citep{GoodfellowDeepLearning}. 
        
The decoder in VAEs either reconstructs the original input or synthesizes new outputs from sampled latent variables. This process is optimized by a loss function that includes both the reconstruction loss and a regularization term based on the Kullback-Leibler (KL) divergence \( D_{\text{KL}} \). This divergence measures the discrepancies between the estimated and true data distributions \citep{kingmaVAE} and improves the model's ability to effectively generalize to unseen data.
        
Through this mechanism, VAEs continuously refine their representation and reconstruction process, improving the generation of new data points that resemble the original training data while maintaining a simplified and structured latent space.

\begin{figure}[ht]
    \centering
      \hspace{.8cm}
      \includegraphics[width=.9\columnwidth]{figures/VAE.png}
      \caption{Functionality of a Variational Autoencoder, demonstrating incorporation of the latent distribution - the mean and standard deviation - for enhancing generative capabilities.}
      \label{fig:figureVAE}
\end{figure}

Despite their capabilities, VAEs exhibit some limitations. According to \citeauthor{GoodfellowDeepLearning}, the generated samples can often be blurry. The reason for this is not fully 
understood, but the blurriness observed may be due to their optimization process, which minimizes Kullback-Leibler divergence. This could lead the model to assign high probabilities to "points that occur in the training set, but may also assign high probability to other points [...] which may include blurry images" \citep{GoodfellowDeepLearning}. The Gaussian distribution often used in VAEs for the generative model may also contribute to this effect, as it can ignore minor features in the input data \citep{GoodfellowDeepLearning}. Another issue is that VAEs typically utilize only a small portion of the latent space, which might further compromise the quality of generated images \citep{GoodfellowDeepLearning}. The performance of the model is also sensitive to the choice of priors for the latent space, making hyperparameter tuning an essential aspect of working with VAEs \citep{kingmaVAE, higginsVAE}. 


\section{Generative Adversarial Networks - GANs}
\label{GAN}

There are two main approaches in Machine learning methods, 
supervised learning and unsupervised learning. Supervised learning requires extensive control and a vast amount of labeled data sets, whereas unsupervised learning offers a more straightforward approach by not requiring classified data to learn a model.

In supervised learning, the ML algorithm learns from a labeled data set where each data point is related to a corresponding target or output value. This enables the algorithm to make predictions or classify new, unseen data based on the patterns it has learned from the labeled examples. The model's predictive capabilities are continuously refined by comparing its outputs to the expected outputs from the training dataset. Through this iterative process, adjustments can be made to enhance the model's performance and generate improved outputs. Yet, it is quite expensive and time-consuming to obtain the large amount of required data as it is often labeled manually by human experts. On the other hand, with the use of generative modeling, Unsupervised learning aims to discover patterns or structures within some unlabeled data. It demonstrates to be particularly useful when labeled data is scarce or unavailable. This concept of unsupervised learning sets the stage for understanding how implicit generative models like GANs address a significant limitation and offer a distinct advantage.

"When a deep neural network is used to generate data, the corresponding density function may be computationally intractable" \citep{goodfellowGAN}. Unlike traditional generative models, implicit generative models do not require the explicit design of a density function to describe the patterns in the data. Instead, they use a sample generation process that produces new samples resembling the existing ones \citep{goodfellowGAN}. Before Generative Adversarial Networks were introduced, the leading implicit generative model was the generative stochastic network, "which is capable of approximately generating samples via an incremental process based on Markov chains" \citep{goodfellowGAN}. Markov chains are a way of describing a sequence of events or states, where probability of transition to the succeeding state is solely dependent on current states. This approach, however,  can be time-intensive and may not always yield accurate results. GANs, on the other hand, directly generate high-quality samples in a single step, overcoming the limitations of incremental generation methods. It is useful to point out that the numerous steps of GAN training refer to iterative updating of model parameters rather than incremental sample generation.

The adversarial aspect of GANs arises from the game-like competition between two neural networks: the generator and the discriminator. The generator is responsible for creating fake inputs or samples, which are then passed to the discriminator. The discriminator's role is to differentiate between real samples from the domain set and the fake samples generated by the generator.

\begin{figure}[ht]
\centering
  \includegraphics[width=1\columnwidth]{figures/Generator.png}
  \caption{Simplified functionality of a Generative Adversarial Network}
  \label{fig:figureGAN}
\end{figure}

Initially, the discriminator is trained on a dataset of unlabeled data, aiming to learn the characteristics and attributes of the desired output. Once it becomes proficient at identifying the genuine objects, it is presented with examples of non-objects and its ability to distinguish these examples is assessed. 
Subsequently, the generator utilizes random input vectors to generate counterfeit versions of the desired objects. 
The discriminator then assesses the authenticity of these outputs and shares the result. Based on this feedback, the generator or the discriminator adjust their behavior. This iterative process, facilitated by an iterator, involves creating samples, updating the models, and repeating the cycle. Gradually, the generator becomes highly skilled at producing realistic outputs that the discriminator can no longer distinguish from real ones. This process is referred to as a zero-sum game, where there is always a winner and a loser. The winner remains unchanged, while the loser updates its model based on the feedback received from the discriminator.

For Images, the Discriminator and the Generator are often implemented as Convolutional Neural Networks (CNNs), which excel at recognizing patterns in images and are commonly used for object identification.
GANs are not limited to images, they can also be applied to tasks such as video frame prediction, image enhancement to improve image quality, and encryption \citep{goodfellowGAN}.

GANs pose a substantial challenge in their training process as they are hard to train \citep{goodfellowGAN}. In addition, \citeauthor{brophyGAN} highlight three important problems commonly associated with GANs, among others. These issues, namely non-convergence, diminishing or vanishing gradients, and mode collapse, contribute to the inherent instability experienced during GAN training. Non-convergence refers to the failure of a GAN model to stabilize and reach a state of equilibrium. Instead, it continuously oscillates and fails to converge to a satisfactory solution. As a result, the model does not learn the underlying patterns of the data and can even diverge, leading to poor performance \citeauthor{brophyGAN}. Diminishing or vanishing gradients occur when the gradients used to update the generator become extremely small or even vanish altogether. This phenomenon is often caused by an overly successful discriminator that becomes too adept at distinguishing real and fake samples. As a result, the generator struggles to learn from the feedback provided by the discriminator, impeding its ability to generate high-quality samples \citeauthor{brophyGAN}. Mode collapse happens when the generator collapses, meaning it focuses on producing only a limited set of samples or outputs, typically lacking diversity and variety \citep{salimansNIPS}. In such cases, the generator fails to capture the full range of patterns and characteristics present in the training data, resulting in uniform and repetitive samples that do not adequately represent the true distribution \citeauthor{brophyGAN}.

\section{Diffusion models}\label{diffusion Models}

The limitations of VAEs and GANs, which were just stated above, have led to the emergence of diffusion models, a method that offer distinct advantages over traditional generative models. Diffusion models operate by progressively perturbing data with noise and then learning to reverse this process to generate new samples. 

~\cite{yangdiffusionSummary} distingueshe between three main approaches that dominate the study of diffusion models, which are going to be discussed shortly: Denoising Diffusion Probabilistic Models (DDPMs) \citep{hoDDPMs,sohlDDPM}, Score-based Generative Models (SGMs) \citep{song2019SGM}, and Stochastic Differential Equations (Score SDEs) \citep{song2020score, song2021maximum}.

\subsection{Denoising Diffusion Probabilistic Models}
DDPMs employ two Markov chains, a forward chain and a reverse chain, also known as the forward and reverse diffusion processes, seen in Figure~\ref{fig:figureForwardProcess} and Figure~\ref{fig:figureReverseProcess} \citep{sohlDDPM}. 

The forward diffusion process is comparable to latent variable models, sharing some similarities with Variational Autoencoders (VAEs), as focus is lying on a latent feature space of the initial data distribution. They differ in the fact that the forward process in DDPMs ``is fixed to a Markov chain that gradually [over a span of T steps] adds Gaussian noise to the data according to a variance schedule \(\beta_1, \ldots, \beta_T \)''~\cite{hoDDPMs}. This iterative process continues to add noise ``until all structures are lost'' \citep{yangdiffusionSummary} resulting in an image of pure noise. The introduction of noise aims to gradually steer the data distribution towards a more manageable prior distribution \citep{yangdiffusionSummary, pooleDreamfusion}. Mathematically, the forward process can be described in two equations:

\[
q(x_t | x_{t-1}) = \mathcal{N}(x_t; \sqrt{1 - \beta_t}x_{t-1}, \beta_t I) \quad \text{where} \quad \sqrt{1 - \beta_t}x_{t-1} = \mu_t \quad \text{and} \quad \beta_t I = \Sigma_t
\] 

\citep{martinez2023understanding}. This describes the process of adding noise to transform a data point \( x_{t-1} \) into a new data point \( x_t \). This transformation is probabilistic and follows a Gaussian distribution \citep{sohlDDPM, hoDDPMs}. The mean value of this distribution is slightly adjusted compared to the previous data point by the factor \( \sqrt{1 - \beta_t} \), which essentially corresponds to the data point \( x_{t-1} \) with a certain noise reduction. The parameter \( \beta_t \) controls the amount of noise added, where a larger \( \beta_t \) means more noise \citep{kingma2023variationalDM}. The covariance matrix, denoted \( \beta_t I \), implies that each element of the data is independently modified with the same amount of noise, since \(I\) represents an identity matrix where all outer diagonal elements are zero \citep{croitoru2023diffusion}.

\[q(x_{1:T} | x_0) = \prod_{t=1}^T q(x_t | x_{t-1}) \] 

\citep{martinez2023understanding}. In the second equation, the focus is on the entire sequence of data points from the original \( x_0 \) to \( x_T \), including all intermediate points \citep{martinez2023understanding}. This expresses the idea that to understand the probability of the entire noisy trajectory, one can calculate the probability of each step from \( x_{t-1} \) to \( x_t \) and then multiply these probabilities together. This is due to the Markov property, which states that each step is only dependent on the immediately preceding step \citep{martinez2023understanding}. This enables a comprehensive view of the transition probabilities over the entire process of noise induction, step by step.

\begin{figure}[ht]
\centering
  \includegraphics[width=1\columnwidth]{figures/manta_DDMP3.png}
  \caption{Forward process adding noise to an image}\label{fig:figureForwardProcess}
\end{figure}

The reverse chain is trained to approximate the inverse of the forward process using a deep neural network parameterized with~\(\Theta\), effectively removing the noise added by the forward chain \citep{sohlDDPM, yangdiffusionSummary}. The function \( p_\theta(x_{t-1} | x_t) \) is a probability distribution that estimates how to reverse the diffusion process. Given a data point \( x_t \), it attempts to predict the previous data point \( x_{t-1} \) before noise was added. It is modeled as a normal distribution with a mean \( \mu_\theta(x_t, t) \) and covariance \( \Sigma_\theta(x_t, t) \), both of which are functions parameterized by \( \theta \) \citep{yangdiffusionSummary}.

\[
  p_\theta(x_{t-1} | x_t) = \mathcal{N}(x_{t-1}; \mu_\theta(x_t, t), \Sigma_\theta(x_t, t))
\] 

\[p_\theta(x_{0:T}) = p_\theta(x_{T}) \prod_{t=1}^T p_\theta(x_{t-1} | x_t) \]

\citep{martinez2023understanding}. The latter function \( p_\theta(x_{0:T}) \) represents the probability of the entire sequence of data points from \( x_0 \) through \( x_T \) under the reverse process modeled by the neural network. It starts with an estimate of the final data point \( p_\theta(x_{T}) \) and works backwards through the sequence, multiplying the conditional probabilities \( p_\theta(x_{t-1} | x_t) \) for each step \citep{hoDDPMs,martinez2023understanding}. This function essentially provides a framework for reconstructing the original data from the noisy data by successively removing noise at each step, based on the learned parameters \( \theta \) \citep{yangdiffusionSummary}.


\begin{figure}[ht]
  \centering
    \includegraphics[width=1\columnwidth]{figures/manta_DDMP3.png}
    \caption{Forward process adding noise to an image}\label{fig:figureReverseProcess}
\end{figure}


The incremental introduction of the forward and backward diffusion processes offers an advantage as ``estimating small perturbations is more tractable than explicitly describing the full distribution with a single, non-analytically-normalizable, potential function'' \citep{sohlDDPM}.
According to \citep{hoDDPMs}, the neural network in the reverse process can be trained to predict one of three possibilities: the mean value of the noise at each time step, the original image itself, or the noise of the image \citep{hoDDPMs}. As previously mentioned, the second approach is not as advantageous. Hence, the research focuses on the first and last possibilities, which are essentially identical but parameterized differently. Predicting the image noise allows for straightforward subtraction of the noise from the image, resulting in a less noisy version. By employing this method iteratively, it becomes possible to completely learn an image from noise.

Nevertheless, DDPMs also have their drawbacks with the most severe being the time requirement to generate new samples \citep{xiao2022tackling}. ``This is caused by the fact that a Markov process has to be simulated at each generation step, which greatly slows down the process'' \citep{martinez2023understanding}. 

\subsection{Score-Based Generative Models}

SGMs \citep{song2019SGM} take a unique approach to generative modeling by prioritizing the learning of a score function that plays a central role in guiding the generative process. This function aims to capture the Stein score \citep{steinScore}, which is essentially ``the gradient of the log-density function at the input data point'' \citep{song2019SGM}. The score can be viewed as a vector field that indicates the direction that increases the data density the most \citep{song2019SGM}. This reveals how slight adjustments within the data can affect the overall distribution, informing the model on how to proceed with sample generation in a way that matches the actual data structure.

To train a neural network for SGMs,~\cite{song2019SGM} employ two techniques, namely Score matching and Langevin dynamics \citep{hyvarinenScoreMatching}. 
Score matching allows to train a neural network, referred to as a score network \( s_\theta(x) \), to predict the gradient of the logarithm of the probability density of the data \( \nabla_x \log p_{\text{data}}(x) \) directly without the need to first construct a model that can estimate the probability density \( p_{\text{data}}(x) \) itself \citep{song2019SGM}. The aim of this process is to minimize the difference between the predictions of the score network and the true gradient of the log-likelihood, which under certain conditions ensures that the trained network approximates the true score ``almost surely'' \citep{song2019SGM}. The term \( \frac{1}{2} \|s_\theta(x)\|^2_2 \) is part of the objective function that needs to be minimized and is used to regularize the scores predicted by the neural network while the term \( tr(\nabla_x s_\theta(x)) \), involving the trace of the Jacobian \citep{song2019SGM} of the score function, further refines the objective function by capturing the divergence of the score vector field. 

\[
\mathbb{E}_{p_{\text{data}}(x)} \left[ \text{tr}(\nabla_x s_\theta(x)) + \frac{1}{2} \|s_\theta(x)\|^2_2 \right]
\]

\citep{song2019SGM} However, scaling score matching to deep networks and high-dimensional data can be challenging due to the high computational effort required to compute certain matrix operations (such as the trace of the Jacobian matrix) \citep{song2019SGM}. 
Denoising score matching and Sliced score matching could overcome these scalability issues for large applications \citep{song2019SGM}.

In order to generate Samples, Langevin dynamics \citep{robertsLangevin} is utilized for sampling from a probability distribution by employing the score function, \( \nabla_x \log p(x) \) \citep{song2019SGM}. The process begins with an initial guess \( \tilde{x}_0 \) from a prior distribution \( \pi(x) \), and iteratively updates this guess according to the rule:

\[ \tilde{x}_t = \tilde{x}_{t-1} + \frac{\epsilon}{2} \nabla_x \log p(\tilde{x}_{t-1}) + \sqrt{\epsilon} z_t, \]

\citep{song2019SGM} where \( z_t \) follows a standard normal distribution and \( \epsilon \) is a small step size. Theoretically, as \( \epsilon \) approaches zero and the number of iterations \( T \) becomes very large, the distribution of \( \tilde{x}_T \) will converge to \( p(x) \) \citep{song2019SGM}. The score network \( s_\theta(x) \) is trained to closely estimate \( \nabla_x \log p_{\text{data}}(x) \), thus allowing for the generation of new samples that approximate the desired data distribution through Langevin dynamics \citep{song2019SGM}.

In the specific context of score-based generative modeling, there's a notable challenge to overcome. The estimated score function may not be entirely accurate in regions of the data space where there's minimal or no training data available \citep{song2019SGM}. In such cases, Langevin Dynamics might not converge correctly, resulting in complications during the sample generation process. To address this issue,~\cite{song2019SGM} suggest ``to perturb the data with random Gaussian noise of various magnitudes'', while simultaneously estimate the score functions for these noise-altered data distributions. This approach is called Noise Conditional Score Networks (NCSNs) and ensures that the resulting data distribution doesn't condense into a lower-dimensional structure \citep{song2019SGM}.

\subsection{Stochastic Differential Equations}

~\cite{song2020score} aim to combine both DDPMs and SGMs (NCSNs) using Stochastic Differential Equations (SDEs) to perturb data across an ``infinite spectrum of noise scales'' \citep{song2020score}. These SDEs are further used for sample generation \citep{yangdiffusionSummary}.

The idea is to create a continuous diffusion process, indexed by time, that transforms a data distribution into a more tractable prior distribution. This is done through the following SDE

\[ dx = f(x, t)dt + g(t)dw, \]

\citep{yangdiffusionSummary} where the data evolves as noise intensity increases over time. The process is governed by two coefficients: a drift coefficient \( f(x, t) \), governing the deterministic properties of the stochastic process, and a diffusion coefficient \( g(t) \), which scales the random noise introduced by Brownian motion \( dw \) (Wiener process) \citep{song2020score}. This Brownian motion represents the random movement of particles in a fluid as they collide with fast-moving molecules in the fluid. 

``A remarkable result from Anderson \citep{anderson1982313} states that the reverse of a diffusion process is also a diffusion process, running backwards in time and given by the [following] reverse-time SDE:\@'' \citep{song2020score}

\[ dx = \left[ f(x, t) - g(t)^2 \nabla_x \log p_t(x) \right] dt + g(t)d\bar{w} \]

This equation describes the process of recovering data from noise by moving backward in time. Knowledge of the score function \( \nabla_x \log p_t(x) \) at each time enables this reverse process, allowing for the generation of data samples from the original distribution using numerical techniques \citep{song2020score}.

\begin{figure}[ht]
  \centering
    \includegraphics[width=1\columnwidth]{figures/DiffusionModels_SDEs.png}
    \caption{Summary of the score-based generative modeling through SDEs \citep{song2020score}}\label{fig:DM_SDEs}
\end{figure}

Knowledge of the scores at each time requires the estimation of the scores for an SDE which involves training a model to approximate the gradient of the log probability of data at various noise levels \citep{song2020score}. The training objective is given by:

\[
\theta^* = \arg\min_\theta \mathbb{E}_t \left\{ \lambda(t) \mathbb{E}_{x_0} \mathbb{E}_{x_t|x_0} \left\| s_\theta(x_t, t) - \nabla_{x_t} \log p_{0t}(x_t | x_0) \right\|_2^2 \right\}
\]

\citep{song2020score} The equation presents a method to find optimal model parameters \( \theta^* \) that minimize the expected discrepancy between the scores estimated by the model and the actual data transition scores over time, which are influenced by the noise \citep{song2020score}. The expectations are taken over time and modulated by a time-varying weighting function \( \lambda(t) \). ``With sufficient data and model capacity, score matching ensures that the optimal solution for the above equation, denoted by \(s_\theta*(x, t)\) equals \(\nabla_{x_(t)} \log p_{t}(x)\) for almost all \( x \) and \( t \)'' \citep{song2020score}. The score matching process matches the output of the score network with the true gradient of the log-likelihood over the course of the SDE, enabling the generation of realistic data samples from complex distributions \citep{yangdiffusionSummary}.



\section{Contrastive Language-Image Pre-training~--~CLIP}\label{CLIP}

\citeauthor{radfordCLIP} address the limitations of traditional computer vision models, which are restricted by their training on a fixed set of object categories and lack adaptability to new tasks or concepts. To overcome these limitations, they propose a novel method called Contrastive Language-Image Pre-training (CLIP), which is an ``efficient and scalable method of learning from natural language supervision''~\citep{radfordCLIP}. This process allows the model to learn a representation of the image that is grounded in natural language, enabling it to understand the content and context of the image.

Unlike traditional computer vision models that rely solely on annotated image datasets, CLIP leverages a large corpus of text and image pairs from the internet. It learns to associate images and their corresponding textual descriptions, allowing it to understand the relationship between visual and textual data. CLIP is built on a transformer-based architecture, which has proven highly effective for natural language processing tasks \citep{radfordCLIP}. It consists of two main components: an image encoder and a language encoder. The image encoder processes images using a modified version of ResNet50 \citep{heResnet} or as a second approach was build upon the Vision Transformer (ViT) \citep{dosovitskiyViT}, while the language encoder uses another modified transformer-based model to process textual descriptions \citep{vaswani2023attention}. By learning to associate images and text, CLIP acquires a generalized understanding of visual concepts and language semantics.

One of the remarkable aspects of CLIP is the ability for zero-shot capability. It can perform tasks without task-specific training. For example, given a natural language prompt, CLIP can recognize objects in images, generate captions, or perform classification tasks. Furthermore, CLIP has been shown to be very effective on a variety of tasks. It outperforms state-of-the-art models on the ImageNet classification task, and it achieves state-of-the-art results on the Visual Genome dataset for object detection and question answering \citep{radfordCLIP}.

However, there exist several limitations to CLIP\@. ``The performance of zero-shot CLIP is often just competitive with the supervised baseline of a linear classifier on ResNet-50 features''~\citep{radfordCLIP}.  This means that CLIP is not significantly better than a model that is trained on labeled data for the specific task at hand. In addition, the authors estimate that achieving SOTA performance across their evaluation suite would require significantly more computational resources, approximately a 1000-fold boost in computational power. Current hardware capabilities cannot accommodate such demands, emphasizing the requirement for advancements in hardware technology to effectively train zero-shot CLIP models.
\section{Multilayer Perceptron~--~MLP}\label{MLP}

Multilayer Perceptrons (MLPs) form a fundamental class of artificial neural networks in machine learning, characterized by their layered structure, including input, hidden, and output layers \citep{noriega2005multilayer, MurtaghMLP}.  Central to their function is the activation function, typically the Logistic Sigmoid Function in MLPs, which transforms the weighted inputs into node outputs in a continuous and differentiable manner, facilitating gradient-based optimization \citep{MurtaghMLP, noriega2005multilayer}. The learning process in MLPs is supervised, involving initial random weight assignment, followed by training through pattern presentation, output comparison, and backward error propagation, typically using gradient descent methods to minimize errors \citep{noriega2005multilayer}. This process is guided by the generalized delta rule or backpropagation, which adjusts network weights based on the error between predicted and actual outputs, allowing the network to effectively learn from its environment \citep{MurtaghMLP}. The architecture of an MLP, including the number of layers and nodes, along with the activation methods and learning techniques, significantly influences its performance. Balancing network complexity and the risk of overfitting is crucial in MLP design \citep{MurtaghMLP}. MLPs are versatile, capable of performing tasks like regression, mapping input vectors to values, and supervised classification, where input patterns are trained to produce specific output classifications \citep{MurtaghMLP}


\section{Representation Forms of 3D Data}\label{3Drepresentation}


\subsection{Meshes, Point-Clouds and Voxels}\label{MPCV}

//TODO
\subsection{Neural Radiance Fields~--~NeRFs}\label{NeRF}

Neural Radiance Fields (NeRFs), as introduced by \citeauthor{mildenhallNERF}, represent a significant advancement in 3D scene representation, particularly when compared to traditional methods which often struggle with complex geometries and varying lighting conditions. Emerging in response to these challenges, NeRFs employ a novel volumetric representation, capturing the spatial and angular distribution of light more accurately \citep{mildenhallNERF}.

\begin{figure}[ht]
    \centering
      \includegraphics[width=1\columnwidth]{figures/NeRF_Fig_2_Mildenhall.png}
      \caption{Sumamrized workflow of a NeRF\@: 5D input (Position + Direction) is processed by the MLP \(F_\theta\) to output color and density. Volume rendering integrates predictions along rays to generate an image from the specified viewpoint. The rendering loss, comparing the rendered and actual images, guides the network's training and refinement process~\cite{mildenhallNERF}}\label{fig:figureNeRF}
\end{figure}

Classic deep learning methods often require a comprehensive dataset comprising various scenes and their representations. NeRFs, on the other hand, are trained to specialize in a single, unique scene \citep{mildenhallNERF}. The underlying neural network, MLP \(F_\theta\), consists of Fully Connected Layers (FCs) with ReLU activations, specifically designed to encode the volumetric details of that particular scene, effectively creating a dedicated neural network for each scene \citep{mildenhallNERF}. 

The neural network accepts two types of input: a position in a given coordinate system, often expressed as a 3D vector \(x, y, z\), and a viewing direction represented by two angles \( \theta, \phi \). To better understand the concept of the latter two inputs, imagine a scenario where a flashlight is held in the middle of a dark room. The angle \( \theta \) would represent how much the flashlight is tilted up or down. Similarly, the angle \( \phi \) would represent how much the flashlight is rotated about the vertical axis while pointed outward. These angles help the neural network understand from which direction the beam is being cast to a point in 3D space. The network's output consists of the color \(c\) and density \( \sigma \) at that particular location \citep{mildenhallNERF}. 


Central to NeRF's operation is the process of volume rendering, which involves casting rays through the scene and gathering color and density information at multiple points.

\[ 
C(r) = \int_{t_n}^{t_f} T(t)\sigma(r(t))c(r(t), d)dt \quad \text{where} \quad T(t) = \exp\left(-\int_{t_n}^t \sigma(r(s))ds\right) 
\]

This formula, as outlined by \citeauthor{mildenhallNERF}, enables the rendering of a 3D scene by casting rays and aggregating their properties from a near boundary (\(t_n\)) to a far boundary (\(t_f\)). The equation essentially builds up the final image by integrating the effects of light interaction with the scene's material over the length of each ray. The volume density \(\sigma\) at a point in the scene indicates the amount of light-blocking material at that specific location. A higher density suggests a greater likelihood of a light ray being obstructed, signifying more material presence at that point. The accumulated transmittance \( T(t) \) reflects the cumulative effect of density along a ray's path. It quantifies the extent to which light from the start of the ray can travel through the scene to a given point without being absorbed or scattered by the material. Its value decreases along the ray's path sa it encounters areas of higher density. Moreover, the term \( c(r(t), d) \) denotes the color at a point along the ray, given the viewing direction \( d \). The interplay of density and transmittance at each point along the ray determines if a ray continues through space or concludes upon encountering an object, with the color at this final point contributing to the rendered image \citep{mildenhallNERF}. 

Volume Rendering enables capturing visual phenomena like lighting, reflections, and transparency, which are often challenging to model with traditional 3D reconstruction techniques. For each pixel in a desired image frame, the neural network is queried at multiple points along a ray projected through the scene to produce a curve representing the density of objects along the ray's path, as seen in Volume Rendering in Figure~\ref{fig:figureNeRF}. The point at which this density curve rises significantly usually corresponds to an object in the scene, and the color at this point is what is rendered for that particular pixel. These density and color curves can be visualized using graphs to illustrate how density and color vary along the ray's path \citep{mildenhallNERF}.

To address multi-view consistency, NeRF predicts color as a function of both location and viewing direction, while density depends solely on location. This separation acknowledges that density, unlike color, remains consistent regardless of viewing angle \citep{hu2023consistentnerf}. 

An integral aspect of setting up NeRFs involves solving the problem of identifying the camera's position and direction for each input image. Methods like Structure-from-Motion (SfM) and Simultaneous Localization and Mapping (SLAM) can address this issue \citep{wei2021nerfingmvs}. Once these parameters are identified, new views can be synthesized by querying the neural network for color and density information along rays projected through the scene \citep{gerats2023dynamic}. 

Training a NeRF model, however, presents its own challenges. Without explicit density data, the model learns by minimizing the loss between predicted and observed values from input images. This is facilitated by the differentiable nature of the entire rendering pipeline, including ray casting, sampling, and color computation \citep{yariv2020multiview}.

Although naive NeRF models may not provide photorealistic results due to the lack of detail, several optimizations have been introduced to improve their performance. One notable improvement is the use of positional encoding techniques that deterministically map 3D coordinates and view directions to a higher dimensional space. This is achieved by using high-frequency features before inputting them to the multilayer perceptron (MLP), which helps optimize the neural radiation fields to better represent high-frequency scene content \citep{mildenhallNERF}. Hierarchical volume sampling is another important optimization. This strategy involves two neural network systems, one coarse and one refined, which are jointly optimized during training. Initially, the coarse network sparsely samples the rays in the 3D scene, and based on this initial sampling, the refined or ``fine'' network is guided to perform a more detailed sampling. This hierarchical approach is critical because dense sampling is very computationally expensive. A two-tier system therefore helps manage computational resources while allowing detailed sampling along rays to obtain the density and color of the sampled points \citep{arandjelović2021nerf}.

One of NeRFs key advantages is its memory efficiency. For example, rendering a single scene with NeRF requires only about five megabytes of memory, which is in stark contrast to voxel grid renderings that require over 15 gigabytes for a comparable scene \citep{mildenhallNERF}. This mismatch in memory requirements underscores NeRF's superior efficiency in terms of data storage and transfer, and represents a compelling advantage over traditional 3D rendering techniques \citep{mildenhallNERF}. Remarkably, the memory requirement of the rendered scene is even smaller than that of the input images, making the model extremely efficient in data storage and transfer \citep{mildenhallNERF}.

However, the NeRF model is not free of limitations. A major challenge is the computational cost associated with training the neural network. The optimization process for a single scene may require about 100,000 to 300,000 iterations, equivalent to a training period of about one to two days, to converge, assuming the use of a single NVIDIA V100 GPU \citep{mildenhallNERF}. While this does not require a data center, it does require a significant amount of time, making NeRF less suitable for scenarios that require rapid implementation. In addition, NeRF rendering is prone to sampling and aliasing issues that can lead to significant artifacts in the synthesized images \citep{rabby2023beyondpixels}. These artifacts arise from the limited sampling of the radiation field, resulting in inaccurate reconstruction of certain features, especially in scenes containing sharp edges or textures \citep{rabby2023beyondpixels}.
\subsection{Deep Marching Tetraheda~--~DMTet}\label{DMTet}

In 3D model generation, there are two primary methods: implicit and explicit 3D representations \citep{shen2021DMTet}. Implicit methods, like Neuronal Radiance Fields (NeRFs) \citep{mildenhallNERF}, use mathematical functions to define shapes, capturing complex details such as the shape's orientation and volume. These representations, while rich in detail, usually require conversion into 3D meshes for practical use. Conversely, explicit methods involve directly defining 3D shapes using specific coordinates, like in the case of meshes. This approach is able to capture and show geometrical details. However, it can be less ideal for computational tasks, especially in machine learning, due to its irregularities \citep{michalkiewicz2019deep}.

DMTet, or Deep Marching Tetrahedra \citep{shen2021DMTet}, emerges as a method representing a pivotal step in converting neural implicit representations into explicit mesh forms. Unlike traditional approaches that often struggle with detailed 3D structures, DMTet leverages advanced algorithms to produce high-fidelity 3D meshes in ``a new differentiable shape representation [\(\ldots\)]'' \citep{shen2021DMTet}. DMTet uniquely combines the benefits of both implicit and explicit 3D representations, leveraging a novel hybrid 3D representation approach.

The method proposed by \citeauthor{shen2021DMTet} produces a deformable and differentiable tetrahedral grid based on a given Sign Distance Field (SDF). A SDF is a function defined over a 3D space, where each point in the space gets a value that represents its shortest distance to a surface. The value is negative if the point is inside an object and positive if it's outside. The first step is therefore always to convert the original input into such a form. Point clouds are transformed directly, while voxels undergo initial conversion into point clouds through sampling points on their surfaces, which can then be used for the calculation. In the tetrahedral grid, each vertex receives an initial predicted SDF value and a deformation offset to represent the surface using an implicit function \citep{shen2021DMTet}. The surface is then refined using subdivision and further ``converted into an explicit [triangular] mesh with a Marching Tetrahedra (MT) algorithm, which we show is differentiable and more performant than the Marching Cube'' \citep{shen2021DMTet}. The final step involves refining the mesh into ``a parameterized surface with a differentiable surface subdivision module'' \citep{shen2021DMTet}.

DMTets allows for supervision on the surface which produces better results in contrast to NeRFs \citep{shen2021DMTet}. Moreover, it is efficient in terms of inference speed and output quality, producing explicit meshes suitable for interactive graphic applications \citep{shen2021DMTet}.
\subsection{Instant Neual Graphics Primitives}\label{InstantNGP}

\citeauthor{M_ller_2022} presents Instant NGP, a technique offering advancements in neural rendering primarily through its efficient encoding techniques and optimized rendering processes. This method is applied across multiple tasks, including Gigapixel Image, Neural Signed Distance Functions (SDF), Neural Radiance Caching (NRC), and Neural Radiance and Density Fields (NeRF) \citep{M_ller_2022}. InstantNGP should not be viewed directly as a distinct representation type for 3D models, but rather as a significant extension of existing methods, especially in the context of NeRFs, which are relevant in this thesis.

InstantNGP utilizes multi-resolution hash encoding, efficiently mapping ``neural network inputs to a higher-dimensional space'' \citep{M_ller_2022}. This process includes hashing, feature vector lookup, linear interpolation, and concatenation with viewing direction parameters, crucial for rendering RGB and density predictions effectively \citep{M_ller_2022}. A significant innovation in InstantNGP is its enhanced ray marching algorithm, combined with an occupancy grid. This grid improves rendering by skipping non-contributing spaces, thus enhancing training times drastically \citep{M_ller_2022}.

The model architecture in InstantNGP varies depending on the task. For NeRFs, the architecture involves two concatenated Multilayer Perceptrons (MLPs), a density MLP, mapping the hash-encoded position to output values for density, and a color MLP, which adds view-dependent color variation, for volumetric representations \citep{M_ller_2022}. Additionally, non-spatial input dimensions like view direction are encoded using techniques like one-blob encoding \citep{oneBlob_mueller} and spherical harmonics \citep{verbin2021refnerf}, enhancing the representations \citep{M_ller_2022}.

InstantNGP opts for concatenation over reduction in its encoding process to maintain the richness of encoded information and facilitate parallel processing of each resolution \citep{M_ller_2022}. However, a challenge is the microstructure due to hash collisions, which leads to a ``grainy'' appearance \citep{M_ller_2022}. Overcoming this issue, potentially through advanced filtering or smoothness techniques, is identified as a key area for further improvement \citep{M_ller_2022}.


