\chapter{Introduction}
\label{ch:introduction}

In today's digital landscape, the demand for 3D models is steadily increasing, driven by the need for immersive and realistic visual experiences. In response, researchers and practitioners have leveraged generative AI techniques to develop innovative methods that can automate the process of creating 3D models. These innovations have the potential to not only reshape the way we interact with digital environments, but also to facilitate the simulation, analysis, and visualization of complex real-world phenomena.

The overall goal of this thesis is to provide a comprehensive examination of advances in automated 3D model generation. Efforts are directed at highlighting the techniques and methodologies that underlie this transformative field. Through a careful comparative study, the capabilities of these technologies are evaluated and questions are raised about their potential for creating 3D models that are characterized by both precision and aesthetic appeal. This research contributes to the ongoing development of computer graphics and the broader field of artificial intelligence.

To provide a foundation for this exploration, a comprehensive examination of the fundamentals of generative AI is undertaken.  Essential generative models such as Variational Autoencoders (VAEs) \citep{kingmaVAE,rezendeVAE}, Generative Adversarial Networks (GANs) \citep{goodfellowGAN} and Diffusion Models \citep{yangdiffusionSummary,hoDDPMs, sohlDDPM} are explained. Additionally, a breif introduction in Contrastive Language-Image Pre-training (CLIP) \citep{radfordCLIP} and various forms of 3D data representation is given. These fundamental concepts provide the necessary foundation for a subsequent in-depth analysis of 3D model generation techniques.

The research also extends to evaluating different approaches to generating 3D models. These approaches include the creation of 3D objects from images, text input, and video sequences, with each approach presenting a unique set of challenges and opportunities. In this thesis, these methods are examined in depth through a comparative study. This investigation is facilitated by well-defined experimental setups, the application of rigorous performance metrics, and the performance of comprehensive results analyses.

In addition, this inquiry addresses future opportunities in the field of 3D modeling. Emerging trends and potential research directions that will redefine the 3D modeling landscape are highlighted. The practical implications of these research findings are carefully considered to highlight their practical application and importance.

//TODO briefly explain test metrics / how evaluated.