\chapter{Introduction}
\label{ch:introduction}

In today's digital landscape, the demand for 3D models is steadily increasing, driven by the need for immersive and realistic visual experiences. Researchers and practitioners have leveraged generative AI techniques to develop innovative methods that automate the process of creating 3D models. These innovations have the potential to reshape how we interact with digital environments and facilitate the simulation, analysis, and visualization of complex real-world phenomena.

Starting out in 3D synthesis can be a challenging experience, particularly for novices with limited prior knowledge. While directly applying the models outlined in Chapter~\ref{ch:models} may appear straightforward, acquiring a more in-depth understanding of the diverse methodologies and their foundational principles greatly enriches the learning process. This deeper insight not only improves the practical application of these models but also opens up possibilities for further advancements in the field of automatic 3D model generation.

This thesis provides a comprehensive examination of advances in automated 3D model generation. It delves into a variety of models and technologies, offering a detailed analysis of their mechanisms, capabilities, and limitations. The study focuses on evaluating the technologies' effectiveness, emphasizing their proficiency in creating functionally robust and aesthetically appealing 3D models. This research contributes significantly to the fields of computer graphics and artificial intelligence, serving as a valuable resource for novices in automatic 3D model generation and inspiring future researchers. The insights gained from this comparative study aim to inspire further innovation in this rapidly evolving field, pushing the boundaries of what is possible in 3D modeling and opening up new avenues of exploration and discovery in 3D synthesis.

To provide a foundation for this exploration, a comprehensive examination of the fundamentals of 2D generative AI is undertaken.  Essential generative models such as Variational Autoencoders (VAEs) \citep{kingmaVAE,rezendeVAE}, Generative Adversarial Networks (GANs) \citep{goodfellowGAN} and Diffusion Models \citep{yangdiffusionSummary,hoDDPMs, sohlDDPM} are explained. Additionally, a brief introduction in Contrastive Language-Image Pre-training (CLIP) \citep{radfordCLIP} and various forms of 3D data representation is given. These fundamental concepts provide the necessary foundation for a subsequent in-depth analysis of 3D model generation techniques.

The research further evaluates different approaches to generating 3D models, including methods based on images, text input, and video sequences. Each method presents unique challenges and opportunities, and they are thoroughly examined through a comparative study. This investigation incorporates well-defined experimental setups and the application of rigorous performance metrics, such as accuracy, efficiency, and realism, to conduct comprehensive results analyses.

Additionally, this thesis addresses future opportunities in the field of 3D modeling. It highlights emerging trends and potential research directions that promise to redefine the 3D modeling landscape. The practical implications of these findings are also considered, underscoring their significance in real-world applications.

By providing a nuanced and detailed exploration of automated 3D model generation, this thesis contributes to the understanding and advancement of this dynamic and impactful field.

//TODO briefly explain test metrics / how evaluated.