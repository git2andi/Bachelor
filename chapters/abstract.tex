\begin{abstract}

The exploration of automatic 3D model generation represents a significant stride in the realm of computer vision and machine learning. This thesis delves into the capabilities of this innovative technology, focusing on a comparative study of various methodologies that facilitate the creation of 3D models from textual and image inputs. The importance of this topic stems from the increasing demand for efficient and accurate 3D model generation in various applications, ranging from virtual reality and gaming to medical imaging.

Despite the advancements in this field, there exists a research gap in comprehensively understanding and comparing different automatic 3D model generation techniques, particularly in the context of their effectiveness, accuracy, and applicability. This thesis aims to bridge this gap by conducting a thorough analysis of selected methods, including DreamFusion \citep{pooleDreamfusion}, Magic3D \citep{lin2023magic3d}, Fantasia3D \citep{chen2023fantasia3d}, Magic123 \citep{qian2023magic123} and Wonder3D \citep{long2023wonder3d}. The research methods include a detailed experimental setup that uses both subjective evaluations and performance metrics to analyze these technologies.
    
The key message of this thesis highlights the differentiated capabilities and limitations of the individual methods and provides insights into their applicability and potential for future development. The results show significant differences in the accuracy and efficiency of the methods examined and highlight the strengths and weaknesses of the individual techniques in different scenarios.
    
By offering a comprehensive comparison of various methodologies in automatic 3D model generation, this thesis not only aids in the understanding of these technologies but also paves the way for future research, particularly in addressing generative biases and exploring emerging trends.

\end{abstract} 
