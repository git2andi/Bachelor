\begin{abstract}
The continued emergence of generative models, deep-learning architectures, and data-driven methods has opened a new era of technological opportunity, particularly in the area of automated 3D model generation. Such advances have become increasingly important in a number of sectors, including gaming, virtual reality, medicine and architecture. Given the increasing demand for 3D objects in these industries, a comprehensive analysis of the underlying technologies is of paramount importance. This thesis attempts to fill this scientific gap by providing a systematic examination of the various methods for automatic 3D model generation.

Based on generative algorithms, machine learning and artificial intelligence, the study examines various techniques and methods that have gained acceptance in this field. The goal is to objectively evaluate their efficiency in creating 3D models that are not only accurate but also visually compelling. This analytical framework is focused on evaluating methods such as Variational Autoencoders (VAEs), Generative Adversarial Networks (GANs), Diffusion Models, and Neural Radiance Fields (NeRFs), among others.

The study employs a carefully designed experimental setup and uses specified performance metrics to quantitatively and qualitatively analyze the strengths and weaknesses of these methods. Through this critical evaluation, the thesis aims to provide both an academic resource and a practical guide for subsequent research and applications in the field of automated 3D model generation. Thus, the contributions of this thesis go beyond a purely academic investigation; it serves as a foundational resource that can both deepen existing understanding and inform future computer graphics methodology.


\end{abstract} 