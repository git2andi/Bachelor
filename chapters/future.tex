\chapter{Future Directions}
\label{ch:future}
The exploration of automatic 3D model generation has revealed a dynamic and rapidly evolving field. While this thesis has covered several significant models, it's important to acknowledge that these represent just a fraction of the ongoing advancements. The beauty of this field lies in its cumulative progress, where each new model not only builds upon existing knowledge but also introduces innovative approaches to enhance the state-of-the-art. This chapter will highlight some of the most promising research trends and potential directions for future exploration, focusing on both computational efficiency and quality enhancement.

The realm of 3D model generation is marked by continuous advancements, with new findings emerging on a near-weekly basis. Although keeping pace with these developments is challenging, certain areas within this domain warrant ongoing attention and improvement. This chapter will also speculate on future possibilities, acknowledging the dynamic nature of this field where today's cutting-edge advancements may soon become tomorrow's standard practices.


\subsection{Emerging Trends in 3D Model Generation}
explore current developments and new directions in the field.
Gaussian Splatting
Lumia AI - Genie


Virtual production is becoming increasingly popular, allowing for more creative and cost-effective backdrops. This rise in demand is expected to continue, driving growth in the market for production-ready 3D assets. The concept of the metaverse is also gaining traction, with significant investments from major companies. This virtual environment is set to redefine experiences ranging from virtual concerts to digital clothing, opening new avenues for 3D modeling applications.

Another emerging trend is the standardization of universal modeling standards. As 3D assets are required to function across various virtual platforms, the need for standardization becomes crucial. Tools like TurboSquid's StemCell are paving the way for more versatile and universally compatible 3D models.

Overall, these trends indicate a shift towards more integrated, AI-driven, and standardized approaches in 3D model generation, pointing towards a future where 3D modeling is more accessible, versatile, and deeply ingrained in various aspects of digital interaction.



\subsection{Potential Research Directions}
suggest areas for future research based on study's findings.
Reduce Computational cost

The future of 3D modeling and additive manufacturing (AM) holds substantial potential for innovation across various sectors. One key area of research is exploring the use of AI in additive manufacturing. With technologies like ChatGPT and Nvidia's AI tools demonstrating the ability to create 3D models from text input, there's a significant opportunity for AI applications to become more prevalent in additive manufacturing.

The healthcare industry is increasingly adopting AM to deliver personalized healthcare solutions. This includes the creation of patient-specific implants and surgical tools customized to individual anatomies. The integration of AM in healthcare facilities for various applications, such as orthopedics, dental, and surgical instrumentation, offers an exciting avenue for research, potentially revolutionizing patient care and treatment.

In industrial markets, the use of metal AM for the production of end-use parts, especially in the aerospace and energy sectors, is experiencing rapid growth. Researching new AM solutions specifically designed for mass production, which combine various printing and finishing technologies, could significantly enhance manufacturing workflows and output

Bioprinting is another promising field, with significant strides being made in using bioprinted human tissue models for drug discovery and development. Researching the ability to simulate human responses to drugs in a laboratory setting using bioprinted models could dramatically streamline drug development processes and potentially eliminate the need for animal testing.

Moreover, there are opportunities to address challenges such as intellectual property protection, managing the complexity of large-scale 3D models, and overcoming adoption barriers due to advanced 3D modeling tools and techniques. These challenges present potential research directions that could significantly impact the field's growth and adoption.
