\chapter{Comparative Study}~\label{ch:comparative study}

This chapter deals with a systematic evaluation of the previously investigated methods for creating 3D models. This analysis is crucial for understanding the real-world applicability and effectiveness of each method. The aim is to compare the theoretical principles with the practical results and to provide a comprehensive evaluation of the performance of each model under experimental conditions.

The chapter begins with a description of the experimental setup that was created to test these models. This includes the specific conditions, frameworks and parameters used to ensure that the comparison is fair and the results reliable. 

The generation process of each model is then presented, showing how each model is derived from its inputs into a 3D model. This also includes measures of performance, including generation time, resource efficiency and versatility of each model, providing a multi-dimensional overview of the capabilities of each method.

Finally, the chapter highlights a thorough analysis of the results obtained from these experiments. This section not only presents the data, but also interprets them in the context of the theoretical background, limitations and potential applications of each model. The insights gained here are helpful for anyone who wants to understand the state of the art in 3D modeling and its practical implications in the real world.

\section{Experimental Setup}\label{Setup}

3D model generation is a demanding task, both in terms of computational power and hardware resources. Each method has unique requirements, which poses a significant challenge in creating a fair baseline for comparative analysis.

To address this, the project Threestudio \citep{threestudio2023} was utilized. This platform provides slightly adapted versions of the official methods, preserving their core functionality while making them more accessible for systems with limited hardware capabilities. Magic123 \citep{qian2023magic123}, for example, was originally tested on a V100 GPU with about 32 GB RAM, but Threestudio's version also works on a T4 GPU with about 15 GB RAM.\@. This adjustment ensures uniform testing conditions across various methods, enabling fair comparisons without the need for high-end hardware. However, this benefit comes at the cost of extended training times and inferior outputs compared to the original versions. Threestudio's specific modifications can be found in their GitHub documentation \citep{threestudio2023}.

Despite the advantages of Threestudio, hardware limitations were still a major problem. The available hardware for this thesis was a single NVIDIA GeForce RTX 2080 GPU with 8 GB of RAM\@. Therefore, Google Colab \citep{googlecolab} was used to mitigate these limitations as it provides access to free GPUs and the ability to run code efficiently.

However, this approach had its own limitations. Colab is limited to a single GPU, whereas most 3D generation methods typically benefit from training with multiple GPUs, to achieve more detailed results and faster computation times. Table 1 shows the differences in training duration for each method, highlighting the gap between personal hardware capabilities and the requirements of the original implementations. Another notable limitation of Google Colab is its unpredictability in terms of how long a notebook can be used before the runtime is reset. Opting for the premium version of Colab can mitigate this issue by providing a more stable runtime, but this comes with a subscription fee.

Threestudio provided a test Colab notebook with an initial implementation for DreamFusion. This setup was further refined and extended by myself to improve its functionality and usability. Changes included the ability to transfer the complete training folder with checkpoints, validation images, configuration details and outputs to Google Drive for easy data access and storage. In addition, this improved notebook now includes comprehensive code snippets for training, refining and exporting for each model beyond the scope of Dreamfusion. To prevent common computational errors due to varying dependencies, additional packages were integrated into the setup.

To unite all the methods used in one notebook, the official implementation of Wonder3D \citep{long2023wonder3d} is also included in the notebook, as well as Evaluate3D, a tool I developed myself for basic geometric comparisons of object files 

It is worth mentioning that the functionality of this notebook is strongly based on Threestudio's guidelines, and as the project evolves, some personal adjustments might be superseded by updates from the authors.



\section{Individual Generation Process}\label{generationProcess}

If not mentioned differently, each model was trained for 10,000 iterations using a single T4 GPU in Google Colab with the High-Ram setting enabled. It is important to note that this hardware configuration does not match the specifications used in the official implementations of these models. Consequently, the results generated in this study may not be as detailed and precise. Despite these limitations, the primary purpose here is to evaluate the basic capabilities of each model, which is possible even with comparatively modest hardware standards.

To effectively demonstrate the generation process of each method, an example prompt was used: ``a robot made of plants''. The choice of this prompt was strategic as concept of a robot is inherently versatile and does not have a rigid definition in terms of appearance or composition. Furthermore, it was assumed that a simple prompt such as ``a robot'' would lead to bland models, reflecting the results observed when applying such a prompt to a 2D diffusion model, specifically Dall-E 3 \citep{dalle3}. To mitigate this, the phrase ``made out of plants'' was added. This not only countered the potential monotony of the models, but also tested the models' ability to represent intricate detail and incorporate color, particularly the various hues associated with plants. The expectation was that this addition would enrich the output of the models and provide a more comprehensive basis for evaluating their detail rendering capabilities.

\textbf{Dreamfusion}~--~The model begins by initializing an object randomly and then refines it throughout the training process. Each prompt initiates the creation of a distinct scene, meaning that even repeated usage of the same text input generates unique objects. This phenomenon is evident in the results presented in Figure~\ref{fig:generationDreamFusion}, and Figure~\ref{fig:secondRobotDreamfusion} included in the Appendix. The entire training process typically spans approximately 1.5 hours, with each iteration consuming a consistent amount of time.

\begin{figure}[ht]
    \centering
    % Subfigure for textual description
    \begin{subfigure}[b]{0.20\textwidth}
        \centering
        \fontsize{9pt}{7pt}\selectfont\text{Iteration = 100}\vspace{3cm}
        \fontsize{9pt}{7pt}\selectfont\text{Iteration = 5000}\vspace{2.85cm}
        \fontsize{9pt}{7pt}\selectfont\text{Iteration = 10000}\vspace{1.95cm}
    \end{subfigure}
    \begin{subfigure}[b]{0.20\textwidth}
        \centering
        \includegraphics[width=\textwidth]{etc/a robot made out of plants/dreamfusion/dreamfusion_plantrobot_1_part2.png}
        \includegraphics[width=\textwidth]{etc/a robot made out of plants/dreamfusion/dreamfusion_plantrobot_5000_part2.png}
        \includegraphics[width=\textwidth]{etc/a robot made out of plants/dreamfusion/dreamfusion_plantrobot_10000_part2.png}
        \caption{}
    \end{subfigure}
    \begin{subfigure}[b]{0.20\textwidth}
        \centering
        \includegraphics[width=\textwidth]{etc/a robot made out of plants/dreamfusion/dreamfusion_plantrobot_1_part1.png}
        \includegraphics[width=\textwidth]{etc/a robot made out of plants/dreamfusion/dreamfusion_plantrobot_5000_part1.png}
        \includegraphics[width=\textwidth]{etc/a robot made out of plants/dreamfusion/dreamfusion_plantrobot_10000_part1.png}
        \caption{}
    \end{subfigure}
    % Subfigure 3
    \hspace{.5cm}
    \begin{subfigure}[b]{0.252\textwidth}
        \centering
        \includegraphics[width=\textwidth]{etc/a robot made out of plants/dreamfusion/dreamfusion_plantrobot_model_resized.png}
        \caption{}
    \end{subfigure}
    \caption{The generation process of Dreamfusion using the prompt ``a robot made out of plants''. Section (c) shows a snapshot of the final mesh generated.}~\label{fig:generationDreamFusion}
\end{figure}

As can be seen in Figure~\ref{fig:generationDreamFusion} parts a and b, the object and its texture are generated simultaneously. The process starts with a small dot that gradually transforms into a more complex shape. At the 100th iteration, the original dot begins to transform into a recognizable shape, and a green hue resembling plant coloration is created. At the 5000th iteration, distinct features such as two legs, a square body and a square head become visible, all retaining the same shade of green. At the last, 10,000th iteration, the model shows a background and small arms sticking out of the robot's body.
Part c of the figure showcases the rendered mesh opened in Meshlab \citep{meshLab}, highlighting alterations made by Threestudio, such as duplicate removal and hole filling. These modifications account for the slight differences between the final mesh and the validation images produced during training. Interestingly, the final mesh lacks detailed plant-like features, one could only assume that the legs kind of resemble moss, but this remains speculation. The mesh primarily shows basic shapes, including a square head, a torso with small protruding rods that could be arms, and large legs. Remarkably, the upper half of the body in the final mesh takes on a yellowish color that differs from the green of the earlier validation images. The reason for this color change remains unclear.

\textbf{Fantasia3D}~--~Generating 3D models can be initialised in multiple ways. The first is to just use the prompt. For this approach, Fantasia3D automatically uses a perfect sphere to initialise its mesh and begins its trainig from there. The second approach is to initially already specify this spheres initial values \([0.5, 0.5, 0.5]\) with differnet ones which would roughly describe the desired object by modifying its values according to \([depth, width, height]\). The last approach would be to initialise the mesh with a custom .obj that already acts as a rough shape input.~\ref{fig:generationFantasia} shows its approach for using nothing but the prompt for generating the plant Robot. 

\begin{figure}[ht]
    \centering
    % Subfigure for textual description
    \begin{subfigure}[b]{0.20\textwidth}
        \centering
        \fontsize{9pt}{7pt}\selectfont\text{Iteration = 0}\vspace{3cm}
        \fontsize{9pt}{7pt}\selectfont\text{Iteration = 5000}\vspace{2.85cm}
        \fontsize{9pt}{7pt}\selectfont\text{Iteration = 10000}\vspace{1.95cm}
    \end{subfigure}
    \begin{subfigure}[b]{0.20\textwidth}
        \centering
        \includegraphics[width=\textwidth]{etc/a robot made out of plants/fantasia3d/fantasia_coarse_robot_0_part2.png}
        \includegraphics[width=\textwidth]{etc/a robot made out of plants/fantasia3d/fantasia_coarse_robot_5000_part2.png}
        \includegraphics[width=\textwidth]{etc/a robot made out of plants/fantasia3d/fantasia_coarse_robot_10000_part2.png}
        \caption{}
    \end{subfigure}
    \begin{subfigure}[b]{0.20\textwidth}
        \centering
        \includegraphics[width=\textwidth]{etc/a robot made out of plants/fantasia3d/fantasia_refine_robot_0_part1.png}
        \includegraphics[width=\textwidth]{etc/a robot made out of plants/fantasia3d/fantasia_refine_robot_5000_part1.png}
        \includegraphics[width=\textwidth]{etc/a robot made out of plants/fantasia3d/fantasia_refine_robot_10000_part1.png}
        \caption{}
    \end{subfigure}
    % Subfigure 3
    \begin{subfigure}[b]{0.37\textwidth}
        \centering
        \includegraphics[width=\textwidth]{etc/a robot made out of plants/fantasia3d/fantasia_plantrobot_model_resized.png}
        \caption{}
    \end{subfigure}
    \caption{a: Fantasia3D starting with only a perfect square and refining this according to the prompt ``a robot made out of plants''. In b: only the appearance of the model gets refined. Image c shows the rendered model }~\label{fig:generationFantasia}
\end{figure}


\begin{figure}[ht]
    \centering
      \includegraphics[width=.2\columnwidth]{etc/a robot made out of plants/fantasia3d/fantasia_refine_robot_kd}
      \includegraphics[width=.2\columnwidth]{etc/a robot made out of plants/fantasia3d/fantasia_refine_robot_roughness}
      \includegraphics[width=.2\columnwidth]{etc/a robot made out of plants/fantasia3d/fantasia_refine_robot_metallic}
      \includegraphics[width=.2\columnwidth]{etc/a robot made out of plants/fantasia3d/fantasia_refine_robot_normal}
      \caption{generated textures from Fantasia3D}~\label{fig:texturesFantasia}
  \end{figure}

\textbf{Magic3D}~--~Test 

\textbf{magic123}~--~Test

\textbf{Wonder3D}~--~Test
\section{Comparative Analysis}\label{comparativeAnalysis}

Conducting a comparative analysis of 3D models presents a challenging endeavor, as the parameters defining their success vary widely based on the intended application. For instance, low-resolution models suffice for real-time rendering in virtual reality or gaming contexts, whereas the film industry demands high-quality renderings. Similarly, in industrial applications, precision down to the millimeter is crucial, necessitating extremely detailed and accurate meshes.

In light of these diverse requirements, this analysis of Section evaluates the generated models across multiple dimensions:

\begin{enumerate}
    \item \textbf{Prompt/Result Fidelity:} Investigates how closely each model aligns with the provided prompt or image. Can the intended subject be identified without prior knowledge of the input?
    \item \textbf{Detail Level:} Examines the intricacy of each model. Are the models finely detailed or do they exhibit a more generalized structure?
    \item \textbf{Texture Realism:} Assesses the authenticity and quality of the model textures. How `real' do they look like?
    \item \textbf{Topology:} Looks into the smoothness or blockiness of the model's surface. Is the topology uniformly smooth or does it exhibit a more fragmented appearance?
    \item \textbf{Model Integrity:} Considers whether the model is a single, cohesive unit or split into multiple segments.
\end{enumerate}

Additionally, technical aspects of each model are analyzed, such as the time required for rendering, efficiency, and resource consumption. Each method is also subjected to specific tests based on particular requirements. For example, when symmetry is a key factor, models are evaluated on their ability to produce symmetric outputs. Similarly, when smoothness is crucial, special attention is given to the topology.

The technical metrics for this analysis are derived from the tensorboard and outputs generated during the training process. The percentages and detailed assessments of geometrical features like symmetry, topology, and the presence of holes in the models are calculated using Evaluate3D. This tool, developed as part of this research, leverages the trimesh Python library \citep{trimesh} to extract detailed insights into the fundamental geometric properties of each 3D model.

In the preceding section, it was observed that the methods yielded diverse outcomes when tasked with a broad prompt, such as creating a `robot made of plants'. The text-to-3D methods faced challenges in initially generating an object, a stark contrast to the image-to-3D methods that benefited from having a reference image, offering some directional guidance and hence a slight advantage. To establish a more leveled playing field for the various methods, the next prompt chosen for testing was ``a high-quality rendering of a Playmobil firefighter''. Playmobil figures are known for their uniform base structures, differing primarily in clothing or texture. This prompt was therefore selected to assess whether a method could accurately capture the fundamental structure dictated by the prompt. The outcomes of each method, applied to this specific prompt, are illustrated in Figure~\ref{fig:resultPlaymobil}.


\begin{figure}[ht]
    \centering
    \small
    \begin{subfigure}[b]{0.18\textwidth}
        \centering
        \includegraphics[width=\textwidth]{etc/a high quality rendering of a playmobil firefighter/dreamfusion/dreamfusion_playmobil_result_resize.png}
        \caption{DreamFusion}
    \end{subfigure}
    \begin{subfigure}[b]{0.179\textwidth}
        \centering
        \includegraphics[width=\textwidth]{etc/a high quality rendering of a playmobil firefighter/magic3d/magic3d_playmobil_result_resize.png}
        \caption{Magic3D}
    \end{subfigure}
    \begin{subfigure}[b]{0.227\textwidth}
        \centering
        \includegraphics[width=\textwidth]{etc/a high quality rendering of a playmobil firefighter/fantasia3d_Magic3DInput/fantasia_playmobil_result_resize.png}
        \caption{Fantasta3D}
    \end{subfigure}
    \begin{subfigure}[b]{0.192\textwidth}
        \centering
        \includegraphics[width=\textwidth]{etc/a high quality rendering of a playmobil firefighter/magic123/magic123_playmobil_result_resize.png}
        \caption{Magic123}
    \end{subfigure}
    \begin{subfigure}[b]{0.181\textwidth}
        \centering
        \includegraphics[width=\textwidth]{etc/a high quality rendering of a playmobil firefighter/wonder3D/wonder3d_playmobil_result_resize.png}
        \caption{Wonder3D}
    \end{subfigure}
    \caption{Results obtained using the prompt ``a high-quality rendering of a Playmobil firefighter''}~\label{fig:resultPlaymobil}
\end{figure}




\begin{figure}[ht]
    \centering
    \small
    \begin{subfigure}[b]{0.25\textwidth}
        \centering
        \includegraphics[width=\textwidth]{etc/Images/playmobil.png}
        \caption{}
    \end{subfigure}
    \begin{subfigure}[b]{0.25\textwidth}
        \centering
        \includegraphics[width=\textwidth]{etc/a high quality rendering of a playmobil firefighter/magic123/magic123_playmobil_refine_back_10000_part1.png}
        \caption{}
    \end{subfigure}
    \begin{subfigure}[b]{0.25\textwidth}
        \centering
        \includegraphics[width=\textwidth]{etc/a high quality rendering of a playmobil firefighter/magic123/magic123_playmobil_coarse_right_10000_part1.png}
        \caption{}
    \end{subfigure}
    \caption{(a) displays the original image for the playmobil figure derived form Dall-E 3; (b and c) show the side and back view of Magic123, resectively}~\label{fig:inputPlaymobil}
\end{figure}


\begin{figure}[ht]
    \centering
    \small
    \begin{subfigure}[b]{0.31\textwidth}
        \centering
        \includegraphics[width=\textwidth]{etc/a rendering of a highly symmetrical loaf of bread/dreamfusion/dreamfusion_bread_result.png}
        \caption{DreamFusion}
        \vspace{0.1cm}
    \end{subfigure}
    \begin{subfigure}[b]{0.31\textwidth}
        \centering
        \includegraphics[width=\textwidth]{etc/a rendering of a highly symmetrical loaf of bread/magic3d/magic3D_bread_result.png}
        \caption{Magic3D}
        \vspace{0.1cm}
    \end{subfigure}
    \begin{subfigure}[b]{0.2\textwidth}
        \centering
        \includegraphics[width=\textwidth]{etc/a rendering of a highly symmetrical loaf of bread/fantasia3d/fantasia_bread_result.png}
        \caption{Fantasta3D}
        \vspace{0.1cm}
    \end{subfigure}

    \begin{subfigure}[b]{0.269\textwidth}
        \centering
        \includegraphics[width=\textwidth]{etc/a rendering of a highly symmetrical loaf of bread/magic123/magic123_bread_result.png}
        \caption{Magic123}
        \vspace{0.1cm}
    \end{subfigure}
    \begin{subfigure}[b]{0.23\textwidth}
        \centering
        \includegraphics[width=\textwidth]{etc/a rendering of a highly symmetrical loaf of bread/wonder3D/wonder3d_bread_result.png}
        \caption{Wonder3D}
        \vspace{0.1cm}
    \end{subfigure}
    \begin{subfigure}[b]{0.23\textwidth}
        \centering
        \includegraphics[width=\textwidth]{etc/Images/bread.png}
        \caption{Original Image}
        \vspace{0.1cm}
    \end{subfigure}
    \caption{Results obtained using the prompt ``a rendering of a highly symmetrical loaf of bread''. Part (f) is the input image for Magic123 and Wonder3D, generated with Dall-E 3}~\label{fig:resultBread}
\end{figure}


\begin{figure}[ht]
    \centering
    \small
    \begin{subfigure}[b]{0.24\textwidth}
        \centering
        \includegraphics[width=\textwidth]{etc/a high-quality rendering of a fern in a wooden pot/dreamfusion/dreamfusion_fern_result.png}
        \caption{DreamFusion}
        \vspace{0.1cm}
    \end{subfigure}
    \begin{subfigure}[b]{0.35\textwidth}
        \centering
        \includegraphics[width=\textwidth]{etc/a high-quality rendering of a fern in a wooden pot/magic3d/magic3d_fern_result.png}
        \caption{Magic3D}
        \vspace{0.1cm}
    \end{subfigure}
    \begin{subfigure}[b]{0.32\textwidth}
        \centering
        \includegraphics[width=\textwidth]{etc/a high-quality rendering of a fern in a wooden pot/fantasia3d/fantasia_fern_result.png}
        \caption{Fantasta3D}
        \vspace{0.1cm}
    \end{subfigure}

    \begin{subfigure}[b]{0.28\textwidth}
        \centering
        \includegraphics[width=\textwidth]{etc/a high-quality rendering of a fern in a wooden pot/magic123/magic123_fern_front_result.png}
        \caption{Magic123}
        \vspace{0.1cm}
    \end{subfigure}
    \begin{subfigure}[b]{0.27\textwidth}
        \centering
        \includegraphics[width=\textwidth]{etc/a high-quality rendering of a fern in a wooden pot/wonder3D/wonder3d_fern_result.png}
        \caption{Wonder3D}
        \vspace{0.1cm}
    \end{subfigure}
    \begin{subfigure}[b]{0.28\textwidth}
        \centering
        \includegraphics[width=\textwidth]{etc/Images/fern.png}
        \caption{Original Image}
        \vspace{0.1cm}
    \end{subfigure}
    \caption{Results obtained using the prompt ``a high-quality rendering of a fern in a wooden pot''.}~\label{fig:resultFern}
\end{figure}


\begin{figure}[ht]
    \centering
    \begin{subfigure}[b]{0.2\textwidth}
        \centering
        \includegraphics[width=\textwidth]{etc/a high-quality rendering of a fern in a wooden pot/magic123/magic123_fern_side_result.png}
        \caption{Magic123}
    \end{subfigure}
    \begin{subfigure}[b]{0.32\textwidth}
        \centering
        \includegraphics[width=\textwidth]{etc/a high-quality rendering of a fern in a wooden pot/wonder3D/rgb_000_right.png}
        \caption{Wonder3D}
    \end{subfigure}
    \caption{The side view of the fern showcasing the limitations of Magic123 in deriving the correct angles}\label{fig:fernSideview}
  \end{figure}



  \begin{figure}[ht]
    \centering
    \small
    \begin{subfigure}[b]{0.295\textwidth}
        \centering
        \includegraphics[width=\textwidth]{etc/a high-quality rendering of a big dog sleeping on a chair/dreamfusion/dreamfusion_dog_front_result.png}
        \caption{DreamFusion}
        \vspace{0.1cm}
    \end{subfigure}
    \begin{subfigure}[b]{0.32\textwidth}
        \centering
        \includegraphics[width=\textwidth]{etc/a high-quality rendering of a big dog sleeping on a chair/magic3d/magic3D_dog_front_result.png}
        \caption{Magic3D}
        \vspace{0.1cm}
    \end{subfigure}
    \begin{subfigure}[b]{0.33\textwidth}
        \centering
        \includegraphics[width=\textwidth]{etc/a high-quality rendering of a big dog sleeping on a chair/fantasia3d/fantasia_dog_front_result.png}
        \caption{Fantasta3D}
        \vspace{0.1cm}
    \end{subfigure}

    \begin{subfigure}[b]{0.267\textwidth}
        \centering
        \includegraphics[width=\textwidth]{etc/a high-quality rendering of a big dog sleeping on a chair/magic123/magic123_dog_front_result.png}
        \caption{Magic123}
        \vspace{0.1cm}
    \end{subfigure}
    \begin{subfigure}[b]{0.27\textwidth}
        \centering
        \includegraphics[width=\textwidth]{etc/a high-quality rendering of a big dog sleeping on a chair/wonder3D/wonder3D_dog_front_result.png}
        \caption{Wonder3D}
        \vspace{0.1cm}
    \end{subfigure}
    \begin{subfigure}[b]{0.28\textwidth}
        \centering
        \includegraphics[width=\textwidth]{etc/Images/dog.png}
        \caption{Original Image}
        \vspace{0.1cm}
    \end{subfigure}
    \caption{Results obtained using the prompt ``a high-quality rendering of a big dog sleeping on a chair''.}~\label{fig:resultDogFront}
\end{figure}


\begin{figure}[ht]
    \centering
    \small
    \begin{subfigure}[b]{0.291\textwidth}
        \centering
        \includegraphics[width=\textwidth]{etc/a high-quality rendering of a big dog sleeping on a chair/dreamfusion/dreamfusion_dog_back_result.png}
        \caption{DreamFusion}
        \vspace{0.1cm}
    \end{subfigure}
    \begin{subfigure}[b]{0.3\textwidth}
        \centering
        \includegraphics[width=\textwidth]{etc/a high-quality rendering of a big dog sleeping on a chair/magic3d/magic3D_dog_back_result.png}
        \caption{Magic3D}
        \vspace{0.1cm}
    \end{subfigure}
    \begin{subfigure}[b]{0.34\textwidth}
        \centering
        \includegraphics[width=\textwidth]{etc/a high-quality rendering of a big dog sleeping on a chair/fantasia3d/fantasia_dog_back_result.png}
        \caption{Fantasta3D}
        \vspace{0.1cm}
    \end{subfigure}

    \begin{subfigure}[b]{0.261\textwidth}
        \centering
        \includegraphics[width=\textwidth]{etc/a high-quality rendering of a big dog sleeping on a chair/magic123/magic123_dog_back_result.png}
        \caption{Magic123}
        \vspace{0.1cm}
    \end{subfigure}
    \begin{subfigure}[b]{0.27\textwidth}
        \centering
        \includegraphics[width=\textwidth]{etc/a high-quality rendering of a big dog sleeping on a chair/wonder3D/wonder3D_dog_back_result.png}
        \caption{Wonder3D}
        \vspace{0.1cm}
    \end{subfigure}
    \caption{Results obtained using the prompt ``a high-quality rendering of a big dog sleeping on a chair''.}~\label{fig:resultDogBack}
\end{figure}




\begin{table}[h]
    \centering
    \small 
    \begin{tabular}{lcccccccc}
    \toprule
    Prompt & Dreamfusion & \multicolumn{2}{c}{Magic3D} & \multicolumn{2}{c}{Fantasia3D} & \multicolumn{2}{c}{Magic123} & Wonder3D \\
    \cmidrule(r){3-4} \cmidrule(lr){5-6} \cmidrule(l){7-8}
    & & \multicolumn{1}{c}{Coarse} & \multicolumn{1}{c}{Refine} & \multicolumn{1}{c}{Geom.} & \multicolumn{1}{c}{Appear.} & \multicolumn{1}{c}{C.} & \multicolumn{1}{c}{R.} &  \\
    \midrule
    Robot & 1:24 & 1:23 & 1:20 & 1:15 & 1:18 & 1:46 & 1:47 & 0:15 \\
    Playmobil & 1:17 & 1:17 & 1:18 & 1:14 & 1:17 & 1:46 & 1:46 & 0:15 \\
    Fern & 1:25 & 1:24 & 1:19 & 1:17 & 1:20 & 1:52 & 1:48 & 0:15 \\
    Bread & 1:25 & 1:21 & 1:21 & 1:17 & 1:20 & 1:54 & 1:52 & 0:15 \\
    \bottomrule
    \end{tabular}
    \caption{Comparison of Generation Times for Different Prompts Across Methods (Hours:Minutes). Legend: C = Coarse, R = Refine, Geom = Geometry, Appear = Appearance.}~\label{table:generation_times_complex}
\end{table}

    
    
    
    