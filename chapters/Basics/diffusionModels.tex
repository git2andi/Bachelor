\section{Diffusion models}
\label{diffusion Models}

The limitations of GANs, which were just stated above, have led to the emergence of diffusion models, a class of AI tools that offer distinct advantages over traditional generative models. These models differ from GANs in their approach to modeling and generating data.  Instead of explicitly modeling the joint distribution of data and noise variables, diffusion models operate by progressively perturbing data with noise and then learning to reverse this process to generate new samples.

\citeauthor{yangdiffusionSummary} distingueshes between three main approaches that dominate the study of diffusion models, which are going to be discussed shortly: Denoising Diffusion Probabilistic Models (DDPMs) \citep{hoDDPMs}, Score-based Generative Models (SGMs) \citep{song2019SGM}, and Stochastic Differential Equations (Score SDEs) \citep{song2020score, song2021maximum}.

\subsection{Denoising Diffusion Probabilistic Models}
DDPMs employ two Markov chains, a forward chain and a reverse chain, also known as the forward and reverse diffusion processes \citep{sohlDDPM}. The forward chain introduces noise to the data, in T steps, using a normal distribution. This iterative process continues until the data converges to a pure isotropic Gaussian noise. The introduction of noise aims to gradually steer the data distribution towards a more manageable prior distribution \citep{yangdiffusionSummary, pooleDreamfusion}. 

\begin{figure}[ht]
\centering
  \includegraphics[width=1\columnwidth]{figures/manta_DDMP3.png}
  \caption{Forward process adding noise to an image}
  \label{fig:figureDDPM}
\end{figure}

The reverse chain, in turn, uses an adaptive parameterized deep neural network to progressively remove the noise added by the forward chain. This iterative process aims to generate data patterns that closely resemble the original data by gradually reducing the noise. Training the reverse chain involves minimizing the Kullback-Leibler (KL) divergence between the joint distributions of the forward and reverse chains and maximizing the variational lower bound (VLB) of the log-likelihood of the data. The KL divergence measures the dissimilarity between two probability distributions, in this case, the forward and reverse chains. By minimizing the KL divergence, the reverse chain is trained to approximate the inverse of the forward process, effectively removing the noise added by the forward chain \citep{sohlDDPM}. The VLB provides a lower bound approximation of the log-likelihood, a measure of how well the model captures the underlying data distribution. Maximizing the VLB ensures that the reverse chain generates data patterns that closely match the original data distribution. \citep{hoDDPMs, sohlDDPM}. It's important to clarify that in the context of diffusion models, VLB is a term associated with variational inference, and its usage might not be directly equivalent to VAEs. Regarding what the reverse network predicts, it typically estimates the parameters of a conditional distribution that represents how to transform the noisy data at each step to recover the original data. This prediction guides the process of removing the noise introduced during the forward chain.

The incremental introduction of the forward and backward diffusion processes offers an advantage as "estimating small perturbations is more tractable than explicitly describing the full distribution with a single, non-analytically-normalizable, potential function" \citep{sohlDDPM}.
According to \citep{hoDDPMs}, the neural network in the reverse process can be trained to predict one of three possibilities: the mean value of the noise at each time step, the original image itself, or the noise of the image \citep{hoDDPMs}. As previously mentioned, the second approach is not as advantageous. Hence, the research focuses on the first and last possibilities, which are essentially identical but parameterized differently. Predicting the image noise allows for straightforward subtraction of the noise from the image, resulting in a less noisy version. By employing this method iteratively, it becomes possible to completely learn an image from noise.


\subsection{Score-Based Generative Models}
SGMs are a class of diffusion models that have gained prominence in the field of generative modeling due to their ability to capture complex data distributions and generate diverse and realistic samples.

Score-based generative models (SGMs) take a unique approach to generative modeling by prioritizing the learning of a score function that plays a central role in guiding the generative process. This score function aims to capture the Stein score \citep{steinScore}, which is essentially "the gradient of the log-density function at the input data point" \citep{song2019SGM}. To put it simply, the score can be seen as a vector field, indicating the direction in which the logarithm of the data density experiences the most significant growth \citep{song2019SGM}. In essence, the score function reveals how the data distribution responds to small variations within the data itself, serving as a guiding principle for the generative model. To train a neural network for SGMs, \cite{song2019SGM} employ a technique called score matching \citep{hyvarinenScoreMatching}. This involves estimating the score function for data points intentionally perturbed with Gaussian noise. This means that score matching effectively learns how to denoise the noisy data and restore the original data distribution. \citep{song2020improved}.

In order to generate Samples, Langevin Dynamics \citep{robertsLangevin} is used, which simulates a particle's movement in a field of potential energy, where the score function serves as the guiding force. By moving data samples along the direction of the score function, Langevin Dynamics effectively 'pulls' them toward areas where data density is higher, essentially places where they blend better with the overall dataset.

However, in the specific context of score-based generative modeling, there's a notable challenge to overcome. The estimated score function may not be entirely accurate in regions of the data space where there's minimal or no training data available \citep{song2019SGM}. In such cases, Langevin Dynamics might not converge correctly, resulting in complications during the sample generation process. 

To address this issue, \cite{song2019SGM} suggest "to perturb the data with random Gaussian noise of various magnitudes", while simultaneously estimate the score functions for these noise-altered data distributions. This approach ensures that the resulting data distribution doesn't condense into a lower-dimensional structure \citep{song2019SGM}.






\subsection{Stochastic Differential Equations}
\textcolor{blue}{Score SDEs provide a flexible framework for modeling and generating complex data distributions with inherent stochasticity. Unlike explicit modeling of the joint distribution of data and noise variables, SDEs focus on describing the dynamics of a diffusion process through differential equations. These equations incorporate both deterministic components that govern the overall trend of the process and stochastic components that account for the inherent randomness in the data generation process.
In Score SDEs, the diffusion process is represented as the continuous-time evolution of a random variable. By specifying the drift and diffusion coefficients within the SDEs, one can capture the behavior and evolution of the data distribution over time. The stochastic nature of SDEs allows for modeling intricate dependencies and capturing complex patterns in the data.
Training SDEs involves estimating the parameters of the drift and diffusion coefficients. This is typically done by maximizing the likelihood of the observed data under the SDE framework. Various estimation techniques, such as maximum likelihood estimation or Bayesian inference, can be employed to optimize the parameters.
SDEs can capture long-term dependencies and accurately model the evolution of data over time. Additionally, the stochastic nature of SDEs enables them to generate diverse and realistic samples by incorporating inherent randomness into the data generation process.}

