\section{Diffusion models}
\label{diffusion Models}

The limitations of VAEs and GANs, which were just stated above, have led to the emergence of diffusion models, a method that offer distinct advantages over traditional generative models. Diffusion models operate by progressively perturbing data with noise and then learning to reverse this process to generate new samples. 

~\cite{yangdiffusionSummary} distingueshe between three main approaches that dominate the study of diffusion models, which are going to be discussed shortly: Denoising Diffusion Probabilistic Models (DDPMs) \citep{hoDDPMs,sohlDDPM}, Score-based Generative Models (SGMs) \citep{song2019SGM}, and Stochastic Differential Equations (Score SDEs) \citep{song2020score, song2021maximum}.

\subsection{Denoising Diffusion Probabilistic Models}
DDPMs employ two Markov chains, a forward chain and a reverse chain, also known as the forward and reverse diffusion processes, seen in Figure~\ref{fig:figureForwardProcess} and Figure~\ref{fig:figureReverseProcess} \citep{sohlDDPM}. 

The forward diffusion process is comparable to latent variable models, sharing some similarities with Variational Autoencoders (VAEs), as focus is lying on a latent feature space of the initial data distribution. They differ in the fact that the forward process in DDPMs "is fixed to a Markov chain that gradually [over a span of T steps] adds Gaussian noise to the data according to a variance schedule \(\beta_1, ..., \beta_T \)" \cite{hoDDPMs}. This iterative process continues to add noise "until all structures are lost" \citep{yangdiffusionSummary} resulting in an image of pure noise. The introduction of noise aims to gradually steer the data distribution towards a more manageable prior distribution \citep{yangdiffusionSummary, pooleDreamfusion}. Mathematically, the forward process can be described in two equations:

\[
q(x_t | x_{t-1}) = \mathcal{N}(x_t; \sqrt{1 - \beta_t}x_{t-1}, \beta_t I) \quad \text{where} \quad \sqrt{1 - \beta_t}x_{t-1} = \mu_t \quad \text{and} \quad \beta_t I = \Sigma_t
\] 

\citep{martinez2023understanding}. This describes the process of adding noise to transform a data point \( x_{t-1} \) into a new data point \( x_t \). This transformation is probabilistic and follows a Gaussian distribution \citep{sohlDDPM, hoDDPMs}. The mean value of this distribution is slightly adjusted compared to the previous data point by the factor \( \sqrt{1 - \beta_t} \), which essentially corresponds to the data point \( x_{t-1} \) with a certain noise reduction. The parameter \( \beta_t \) controls the amount of noise added, where a larger \( \beta_t \) means more noise \citep{kingma2023variationalDM}. The covariance matrix, denoted \( \beta_t I \), implies that each element of the data is independently modified with the same amount of noise, since \(I\) represents an identity matrix where all outer diagonal elements are zero \citep{croitoru2023diffusion}.

\[q(x_{1:T} | x_0) = \prod_{t=1}^T q(x_t | x_{t-1}) \] 

\citep{martinez2023understanding}. In the second equation, the focus is on the entire sequence of data points from the original \( x_0 \) to \( x_T \), including all intermediate points \citep{martinez2023understanding}. This expresses the idea that to understand the probability of the entire noisy trajectory, one can calculate the probability of each step from \( x_{t-1} \) to \( x_t \) and then multiply these probabilities together. This is due to the Markov property, which states that each step is only dependent on the immediately preceding step \citep{martinez2023understanding}. This enables a comprehensive view of the transition probabilities over the entire process of noise induction, step by step.

\begin{figure}[ht]
\centering
  \includegraphics[width=1\columnwidth]{figures/manta_DDMP3.png}
  \caption{Forward process adding noise to an image}
  \label{fig:figureForwardProcess}
\end{figure}

The reverse chain is trained to approximate the inverse of the forward process using a deep neural network parameterized with~\(\Theta\), effectively removing the noise added by the forward chain \citep{sohlDDPM, yangdiffusionSummary}. The function \( p_\theta(x_{t-1} | x_t) \) is a probability distribution that estimates how to reverse the diffusion process. Given a data point \( x_t \), it attempts to predict the previous data point \( x_{t-1} \) before noise was added. It is modeled as a normal distribution with a mean \( \mu_\theta(x_t, t) \) and covariance \( \Sigma_\theta(x_t, t) \), both of which are functions parameterized by \( \theta \) \citep{yangdiffusionSummary}.

\[
  p_\theta(x_{t-1} | x_t) = \mathcal{N}(x_{t-1}; \mu_\theta(x_t, t), \Sigma_\theta(x_t, t))
\] 

\[p_\theta(x_{0:T}) = p_\theta(x_{T}) \prod_{t=1}^T p_\theta(x_{t-1} | x_t) \]

\citep{martinez2023understanding}. The latter function \( p_\theta(x_{0:T}) \) represents the probability of the entire sequence of data points from \( x_0 \) through \( x_T \) under the reverse process modeled by the neural network. It starts with an estimate of the final data point \( p_\theta(x_{T}) \) and works backwards through the sequence, multiplying the conditional probabilities \( p_\theta(x_{t-1} | x_t) \) for each step \citep{hoDDPMs,martinez2023understanding}. This function essentially provides a framework for reconstructing the original data from the noisy data by successively removing noise at each step, based on the learned parameters \( \theta \) \citep{yangdiffusionSummary}.


\begin{figure}[ht]
  \centering
    \includegraphics[width=1\columnwidth]{figures/manta_DDMP3.png}
    \caption{Forward process adding noise to an image}
    \label{fig:figureReverseProcess}
\end{figure}


The incremental introduction of the forward and backward diffusion processes offers an advantage as "estimating small perturbations is more tractable than explicitly describing the full distribution with a single, non-analytically-normalizable, potential function" \citep{sohlDDPM}.
According to \citep{hoDDPMs}, the neural network in the reverse process can be trained to predict one of three possibilities: the mean value of the noise at each time step, the original image itself, or the noise of the image \citep{hoDDPMs}. As previously mentioned, the second approach is not as advantageous. Hence, the research focuses on the first and last possibilities, which are essentially identical but parameterized differently. Predicting the image noise allows for straightforward subtraction of the noise from the image, resulting in a less noisy version. By employing this method iteratively, it becomes possible to completely learn an image from noise.

Nevertheless, DDPMs also have their drawbacks with the most severe being the time requirement to generate new samples \citep{xiao2022tackling}. "This is caused by the fact that a Markov process has to be simulated at each generation step, which greatly slows down the process" \citep {martinez2023understanding}. 

\subsection{Score-Based Generative Models}
SGMs are a class of diffusion models that have gained prominence in the field of generative modeling due to their ability to capture complex data distributions and generate diverse and realistic samples.

Score-based generative models (SGMs) take a unique approach to generative modeling by prioritizing the learning of a score function that plays a central role in guiding the generative process. This score function aims to capture the Stein score \citep{steinScore}, which is essentially "the gradient of the log-density function at the input data point" \citep{song2019SGM}. To put it simply, the score can be seen as a vector field, indicating the direction in which the logarithm of the data density experiences the most significant growth \citep{song2019SGM}. In essence, the score function reveals how the data distribution responds to small variations within the data itself, serving as a guiding principle for the generative model. To train a neural network for SGMs, \cite{song2019SGM} employ a technique called score matching \citep{hyvarinenScoreMatching}. This involves estimating the score function for data points intentionally perturbed with Gaussian noise. This means that score matching effectively learns how to denoise the noisy data and restore the original data distribution. \citep{song2020improved}.

In order to generate Samples, Langevin Dynamics \citep{robertsLangevin} is used, which simulates a particle's movement in a field of potential energy, where the score function serves as the guiding force. By moving data samples along the direction of the score function, Langevin Dynamics effectively 'pulls' them toward areas where data density is higher, essentially places where they blend better with the overall dataset.

However, in the specific context of score-based generative modeling, there's a notable challenge to overcome. The estimated score function may not be entirely accurate in regions of the data space where there's minimal or no training data available \citep{song2019SGM}. In such cases, Langevin Dynamics might not converge correctly, resulting in complications during the sample generation process. 

To address this issue, \cite{song2019SGM} suggest "to perturb the data with random Gaussian noise of various magnitudes", while simultaneously estimate the score functions for these noise-altered data distributions. This approach ensures that the resulting data distribution doesn't condense into a lower-dimensional structure \citep{song2019SGM}.






\subsection{Stochastic Differential Equations}
\textcolor{blue}{Score SDEs provide a flexible framework for modeling and generating complex data distributions with inherent stochasticity. Unlike explicit modeling of the joint distribution of data and noise variables, SDEs focus on describing the dynamics of a diffusion process through differential equations. These equations incorporate both deterministic components that govern the overall trend of the process and stochastic components that account for the inherent randomness in the data generation process.
In Score SDEs, the diffusion process is represented as the continuous-time evolution of a random variable. By specifying the drift and diffusion coefficients within the SDEs, one can capture the behavior and evolution of the data distribution over time. The stochastic nature of SDEs allows for modeling intricate dependencies and capturing complex patterns in the data.
Training SDEs involves estimating the parameters of the drift and diffusion coefficients. This is typically done by maximizing the likelihood of the observed data under the SDE framework. Various estimation techniques, such as maximum likelihood estimation or Bayesian inference, can be employed to optimize the parameters.
SDEs can capture long-term dependencies and accurately model the evolution of data over time. Additionally, the stochastic nature of SDEs enables them to generate diverse and realistic samples by incorporating inherent randomness into the data generation process.}
