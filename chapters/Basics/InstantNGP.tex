\subsection{Instant Neual Graphics Primitives}\label{InstantNGP}

\citeauthor{M_ller_2022} presents Instant NGP, a technique offering advancements in neural rendering primarily through its efficient encoding techniques and optimized rendering processes. This method is applied across multiple tasks, including Gigapixel Image, Neural Signed Distance Functions (SDF), Neural Radiance Caching (NRC), and Neural Radiance and Density Fields (NeRF) \citep{M_ller_2022}. InstantNGP should not be viewed directly as a distinct representation type for 3D models, but rather as a significant extension of existing methods, especially in the context of NeRFs, which are relevant in this thesis.

InstantNGP utilizes multi-resolution hash encoding, efficiently mapping ``neural network inputs to a higher-dimensional space'' \citep{M_ller_2022}. This process includes hashing, feature vector lookup, linear interpolation, and concatenation with viewing direction parameters, crucial for rendering RGB and density predictions effectively \citep{M_ller_2022}. A significant innovation in InstantNGP is its enhanced ray marching algorithm, combined with an occupancy grid. This grid improves rendering by skipping non-contributing spaces, thus enhancing training times drastically \citep{M_ller_2022}.

The model architecture in InstantNGP varies depending on the task. For NeRFs, the architecture involves two concatenated Multilayer Perceptrons (MLPs), a density MLP, mapping the hash-encoded position to output values for density, and a color MLP, which adds view-dependent color variation, for volumetric representations \citep{M_ller_2022}. Additionally, non-spatial input dimensions like view direction are encoded using techniques like one-blob encoding \citep{oneBlob_mueller} and spherical harmonics \citep{verbin2021refnerf}, enhancing the representations \citep{M_ller_2022}.

InstantNGP opts for concatenation over reduction in its encoding process to maintain the richness of encoded information and facilitate parallel processing of each resolution \citep{M_ller_2022}. However, a challenge is the microstructure due to hash collisions, which leads to a ``grainy'' appearance \citep{M_ller_2022}. Overcoming this issue, potentially through advanced filtering or smoothness techniques, is identified as a key area for further improvement \citep{M_ller_2022}.