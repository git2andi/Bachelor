\subsection{Deep Marching Tetraheda~--~DMTet}\label{DMTet}

In 3D model generation, there are two primary methods: implicit and explicit 3D representations \citep{shen2021DMTet}. Implicit methods, like Neuronal Radiance Fields (NeRFs) \citep{mildenhallNERF}, use mathematical functions to define shapes, capturing complex details such as the shape's orientation and volume. These representations, while rich in detail, usually require conversion into 3D meshes for practical use. Conversely, explicit methods involve directly defining 3D shapes using specific coordinates, like in the case of meshes, voxels, or point clouds. This approach is able to capture and show geometrical details. However, it can be less ideal for computational tasks, especially in machine learning, due to its irregularities \citep{michalkiewicz2019deep}.

DMTet, or Deep Marching Tetrahedra \citep{shen2021DMTet}, emerges as a method representing a pivotal step in converting neural implicit representations into explicit mesh forms. Unlike traditional approaches that often struggle with detailed 3D structures, DMTet leverages advanced algorithms to produce high-fidelity 3D meshes in ``a new differentiable shape representation [\(\ldots\)]'' \citep{shen2021DMTet}. DMTet uniquely combines the benefits of both implicit and explicit 3D representations, leveraging a novel hybrid 3D representation approach.

The method propesed by \citeauthor{shen2021DMTet} produces a deformable and differentiable tetrahedral grid based on a given Sign Distance Field (SDF). A SDF is a function defined over a 3D space, where each point in the space gets a value that represents its shortest distance to a surface. The value is negative if the point is inside an object and positive if it's outside. The first step is therefore always to convert the original input into such a form. Point clouds are transformed directly, while voxels undergo initial conversion into point clouds through sampling points on their surfaces, which can then be used for the calculation. In the tetrahedral grid, each vertex receives an initial predicted SDF value and a deformation offset to represent the surface using an implicit function \citep{shen2021DMTet}. The surface is then refined using subdivision and further ``converted into an explicit [triangular] mesh with a Marching Tetrahedra (MT) algorithm, which we show is differentiable and more performant than the Marching Cube'' \citep{shen2021DMTet}. The final step involves refining the mesh into ``a parameterized surface with a differentiable surface subdivision module'' \citep{shen2021DMTet}.

DMTets allows for supervision on the surface which produces better results in contrast to NeRFs \citep{shen2021DMTet}. Moreover, it is efficient in terms of inference speed and output quality, producing explicit meshes suitable for interactive graphic applications \citep{shen2021DMTet}.