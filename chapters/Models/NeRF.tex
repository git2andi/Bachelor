\section{Neural Radiance Fields - NeRFs}
\label{NeRF}

\cite{mildenhallNERF, tancikNerfstudio} 

\citeauthor{mildenhallNERF} introduces a new approach to representing 3D scenes from a collection of 2D images. The proposed method, called Neural Radiance Fields (NeRFs), is a volumetric representation that captures both the spatial and angular distribution of light in a scene. The goal of view synthesis is to generate novel views of a scene given a limited number of images from different viewpoints.

Traditional approaches for view synthesis rely on 3D reconstruction techniques, which estimate the geometry and appearance of the scene from multiple images. However, these methods often suffer from artifacts and inaccuracies, especially when the scene has complex geometry or lighting conditions. NeRF addresses these limitations by directly modeling the scene's volumetric representation as a continuous 5D function, called the Neural Radiance Field. This function represents both the 3D spatial coordinates and the viewing direction as inputs, and outputs the scene's color and opacity at that point.

To train the NeRF model, the authors propose a novel method that combines supervised learning and differentiable rendering. They create a synthetic training dataset by generating virtual scenes and capturing corresponding images from known viewpoints. The NeRF model is then trained to optimize the rendering loss, which measures the discrepancy between the synthesized images and the ground truth images.During inference, given a novel view pose, NeRF can estimate the corresponding color and opacity by querying the learned neural network. This enables the synthesis of photo-realistic images from arbitrary viewpoints within the scene.

NeRF are effective on a variety of challenging scenes, including synthetic and real-world datasets. The synthesized images generated by NeRF exhibit high-quality visual details, accurate geometry, and realistic lighting effects. NeRF outperforms traditional methods in terms of view synthesis quality and generalization to novel viewpoints.

However, NeRF has some limitations. It requires a significant amount of computational resources and time for training due to the complex 5D representation and the large amount of data required. Additionally, NeRF struggles with handling dynamic scenes or scenes with moving objects since it assumes a static scene during training. To overcome these limitations, the authors propose extensions to NeRF, such as NeRF in the wild, which aims to handle more diverse and dynamic scenes, and Neural Radiance Caching, which enhances the efficiency of NeRF inference.
