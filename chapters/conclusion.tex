\chapter{Conclusion}~\label{ch:conclusion}
//TODO

\section{Summary of Findings}

This thesis has explored various methods used for automatic 3D model generation, focusing on DreamFusion, Magic3D, Fantasia3D, Magic123, and Wonder3D. Each method was not only introduced but also scrutinized through a comparative analysis. The examination revealed that DreamFusion, one of the earlier methods, tended to produce more blurred outputs compared to its successors. This could be attributed to the initial stages of development in this field. In contrast, newer models like Fantasia3D showed improvements, especially when initiated from a shape approximating the target object. Interestingly, a random sphere as a starting point was found ineffective.

A significant aspect noted was the evolution in speed and efficiency of model generation. Earlier methods required more time for results that aligned with the text prompts, whereas newer models showcased rapid generation capabilities. Despite advancements, challenges in generating intricate details, such as plants, persisted across all methods, highlighting areas for future optimization. Fantasia3D, with its advanced texture generation stage, excelled in producing realistic textures, though it struggled in overall model generation.

One of the most promising developments observed was in the newest methods, such as Genie, which demonstrated fast and detailed outputs, signaling rapid progress in the field. However, the presence of bias in generative AI was a recurring concern, underscoring the need for continued research and development to address this issue.



In a series of tasks testing the capabilities of various 3D rendering methods, distinct patterns emerged across different prompts. For the Playmobil figure, characterized by its uniform base structure, DreamFusion struggled with intricate geometry but maintained color accuracy, while Magic3D excelled in realistic textures and geometric precision for simpler, flat structures. Fantasta3D often overemphasized realism at the expense of essential prompt features, and Magic123, although capturing the essence well, added superfluous elements. Wonder3D frequently failed in even basic details. In the bread prompt, emphasizing symmetry and specific properties, DreamFusion continued to falter with structure, Magic3D showed robustness to directive prompts, and Fantasia3D was disoriented by specificity. Magic123's replication of the input image raised questions about its adaptability, and Wonder3D struggled with overall structure.

When rendering a sleeping dog, a task focusing on organic forms, texture, and anatomy, DreamFusion displayed the lowest detail level, failing to distinguish the subject. Magic3D effectively captured essential elements but lacked textural depth. Fantasia3D, focusing on realistic textures, missed basic geometric accuracy. Magic123 accurately shaped the original image but struggled with different views, while Wonder3D provided a consistent, multi-view output. Finally, in rendering a fern with complex elements and contrasting materials, DreamFusion captured basic shapes but missed finer details. Magic3D's output was disappointing for intricate objects, whereas Fantasia3D achieved high textural realism but lacked geometric accuracy. Magic123's results were basic and view-dependent, and Wonder3D, although capturing the overall form, revealed undetailed textures and floating parts upon closer examination. Across these prompts, each method displayed unique strengths and weaknesses, highlighting the challenges and potentials in automatic 3D model generation.

\section{Contributions to the Field}

This thesis has significantly contributed to the field of automatic 3D model generation. It has elucidated the basics and detailed the functionality of key methods, making the complex domain accessible to a broader audience. The work showcases realistic expectations of model generation for average users, distinct from the high-end results often illustrated in official papers, which require substantial computational resources.

A practical aspect of this thesis is the provision of a notebook facilitating the application of these methods, enabling users to assess and validate the capabilities of each method personally. Additionally, a comprehensive comparison across various methods offers insights into their strengths and weaknesses. The inclusion of Evaluate3D in the notebook provides a user-friendly tool for analyzing generated models, further enhancing the practical utility of this thesis in the field.

\section{Implications and Practical Applications}
discusses real-world impact of the findings.