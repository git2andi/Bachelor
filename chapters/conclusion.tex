\chapter{Conclusion}~\label{ch:conclusion}

This work has shown that generative 2D methods are essential for the creation of 3D models, as they form the basis for better results in guiding the generation process, which also underlines the importance of understanding these methods in order to fully grasp the complexity of automatic 3D synthesis.

Through an in-depth investigation, this thesis has highlighted a number of techniques for the automatic generation of 3D models, focusing on text-to-3D models such as DreamFusion, Magic3D and Fantasia3D, and image-to-3D methods such as Magic123 and Wonder3D. Each of these methods brings a unique approach to the field. DreamFusion, a pioneering approach, has significantly influenced subsequent research with its integration of SDS gradients and diffusion models, enabling NeRF generation without reliance on pre-existing 3D datasets. Magic3D has further advanced this area with its efficient coarse-to-fine optimization, resulting in improved output quality. Fantasia3D differentiates itself by separating geometry and appearance, yielding high-quality textures adaptable to specific input shapes. In contrast, Magic123 merges 2D and 3D priors for a comprehensive generation process, blending textural variety with precise geometric detailing. Finally, Wonder3D applies an innovative strategy by generating color images and normal maps from different viewpoints and then using them to generate consistent 3D models from multiple viewpoints.

The comparative study within the thesis also shed light on the practical aspects of these methods, such as the differences in computational resource requirements between project Threestudio and standard code implementations. The comparison was driven by a series of prompts, each designed to evaluate the capabilities of the methods in question. These assessments centered on how accurately each method responded to the given prompts and the fidelity of the resulting objects. This evaluation process delved into analyzing the basic structure and geometry of each method, comparing their effectiveness in this regard. The final aspect of the assessment focused on determining the extent to which a method could convincingly replicate real-world textures.

\section{Summary of Findings}

In the comprehensive evaluation of various 3D model generation methods, several critical findings emerged, highlighting the diverse capabilities and challenges faced by these techniques. 

DreamFusion exhibits commendable performance in color fidelity, making it suitable for tasks where color representation is a priority. However, it encounters significant challenges in accurately rendering detailed geometries. This method tends to simplify complex structures, which becomes particularly problematic in scenarios requiring high structural detail. As a result, DreamFusion often produces models that are less accurate and overly smooth, lacking the intricate details that might be essential for certain applications.

Magic3D distinguishes itself in rendering realistic textures and achieving geometric accuracy. Its strength lies especially in handling simpler structures and responding well to clear and direct prompts. This method demonstrates a robust ability to interpret and execute specific instructions, making it highly effective for tasks where precision in texture and shape is required. However, its performance might be limited when dealing with more complex or abstract prompts, where the clarity of instructions is less defined.

Fantasia3D aims for high texture realism but tends to overcomplicate models by adding details that are not necessary. This approach often leads to a disparity between the model's complexity and the intended simplicity of the prompt. While this method shows potential in adding intricate textures, the tendency to deviate from the core requirements of the prompt can result in models that are more complex than needed, potentially obscuring the intended design or concept.

Magic123 generally succeeds in capturing the essence of the prompts but also has a tendency to introduce extraneous elements into its outputs. This method manages to grasp the main concept of the prompt and consistently produces solid results. While this approach can add interesting dimensions to the models, it may also lead to a divergence from the original intent of the design, especially in scenarios where simplicity and adherence to the prompt are crucial.

Wonder3D is characterized by its struggle to include basic details in models, often resulting in outputs that appear unfinished or incomplete. This method faces challenges in accurately capturing essential elements of the prompts, which is a significant drawback for applications requiring high levels of detail and completeness. While it may maintain multi-view shape and texture consistency, the lack of detailed texturing and finishing touches signifies a need for improvement in its approach to model generation.

The evaluation of 3D model generation methods highlighted several key findings. Technical assessments revealed high generation times across all methods, emphasizing a need for efficiency improvements. Quantitative metrics like CLIP scores and symmetry evaluations showed that achieving reliable and high-quality outputs remains challenging, indicating a significant research gap in developing more effective evaluation metrics for these technologies.

Looking forward, the field is expanding beyond traditional methods, exploring new avenues such as video-to-3D transformations and leveraging tools like LumaAI's Genie for accessible, high-quality modeling with lower computational demands. However, this progress brings a responsibility to address potential biases within these methods, underscoring the importance of ethical considerations and responsible use to prevent negative impacts.

\section{Contributions to the Field}

This thesis expands the understanding of automatic 3D model generation. It introduces basic concepts and explains the operation of key methods to make this complex field accessible to a wider audience. By contrasting basic methods with advanced techniques, the work clarifies a complex field, making it more accessible. This work sets realistic expectations for the average user regarding model generation, contrasting the often high-end, computationally intensive results depicted in official publications. Notable contributions include a thorough comparative analysis of the leading methods, highlighting their strengths, weaknesses and applications. A key achievement is the development of a user-friendly notebook, enabling practical engagement with these technologies. The inclusion of Evaluate3D in the notebook enhances its practical value, offering an intuitive platform for analyzing generated models. This integration underscores the thesis's commitment to bridging theoretical knowledge and practical application. 
