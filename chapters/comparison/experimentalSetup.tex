\section{Experimental Setup}\label{Setup}

3D model generation is a demanding task, both in terms of computational power and hardware resources. Each method has unique requirements, which poses a significant challenge in creating a fair baseline for comparative analysis.

To address this, the GitHub project Threestudio \citep{threestudio2023} was used. This platform provides slightly adapted versions of the official methods, preserving their core functionality while making them more accessible for systems with limited hardware capabilities. Magic123 \citep{qian2023magic123}, for example, was originally tested on a V100 GPU with about 32GB RAM, but Threestudio's version also works on a T4 GPU with about 15GB RAM\@. This adjustment ensures uniform testing conditions across various methods, enabling fair comparisons without the need for high-end hardware. However, this benefit comes at the cost of extended training times and inferior outputs compared to the original versions. Threestudio's modifications can be found in the Appendix Chapter~\ref{ch:differences}. 

Despite the advantages of Threestudio, hardware limitations were still a major problem. The available hardware for this thesis was a single NVIDIA GeForce RTX 2080 GPU with 8GB of RAM\@. Therefore, Google Colab \citep{googlecolab} was used to mitigate these restrictions as it provides access to free GPUs (with about 15GB RAM) and the ability to run code efficiently. However, this approach had its own limitations. Colab is limited to a single GPU, whereas most 3D generation methods typically benefit from training with multiple GPUs, to achieve more detailed results and faster computation times. Another notable drawback of Google Colab is its unpredictability in terms of how long a notebook can be used before the runtime is reset. Opting for the premium version of Colab can mitigate this issue by providing a more stable runtime, but this comes with a subscription fee.

Threestudio provided a test Colab notebook with an initial implementation for DreamFusion. This setup was further refined and extended by myself to improve its functionality and usability. Changes included the ability to transfer the complete training folder with checkpoints, validation images, configuration details and outputs to Google Drive for easy data access and storage. In addition, this improved notebook now includes comprehensive code snippets for training, refining and exporting for each model beyond the scope of Dreamfusion. To prevent common computational errors due to varying dependencies, additional packages were integrated into the setup. 
Additionally, a slightly revised Notebook for the official Wonder3D implementation \citep{long2023wonder3d} is also given, as well as Evaluate3D, a tool I developed myself for basic geometric comparisons of object files. This tool also uses some functions of CLIP to generate a CLIP score for each object to evaluate its accuracy \citep{radfordCLIP}.

It is worth mentioning that the functionality of this notebook is strongly based on Threestudio's and Wonder3D'S guidelines, and as the project evolves, some personal adjustments might become unnecessary due to updates by the original authors.